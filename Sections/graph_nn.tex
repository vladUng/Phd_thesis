\subsubsection{Graph Neural Networks}

Combining multiple data types bears different names depending on the field ranging from integrative datasets, multi-omics or multi-modal. In the context of Graph Neural Networks (GNN), most commonly the multi-modal models/learning is used when different modals (or data types) are combined (or fusion) to do some sort of classification or clustering. A common example, is that we, humans, have a multi-modal learning by integrating the different sensorial stimuli. Multi-modal should not be confused with multi-view which referes to same data type (modal) but with different data sources (views). For example, the visual experience is composed of the colours, textures and other aspects of the objects \cite{Eman_Alshari_undated-vv}.

I believe the multi-omics problem can be seen both as multi-modal and multi-view. As we are aiming to integrate healthy with tumour datasets (i.e., multi-view) with mutations (i.e., multi-modal).

A very interesting (and unexpected) research area are the Natural Language Processing (NLP) models which may be worth exploring. My understanding (so far) is that when you're trying to translate sentences to a different language, you need to take in account the context and know how the words relate to each other in a sentence. This might be the same problem as finding the the marker genes, where you have TPMs similar and you want to group this together. It's an interesting way to encode information, but the problem with this approach is that you need lots of data.