\thispagestyle{plain}
\begin{center}
    \Large        
    \textbf{Abstract}
    \vspace{0.2cm}
\end{center}


% The aim of the project
Traditional stratification methods for muscle-invasive bladder cancer (MIBC) stratification often rely on a single omics approach, such as gene expression or mutation profiles, to derive subtypes. This project introduces a novel co-expressed network pipeline that allows the integration of multiple data types at multiple levels of the networks, with the goal of improving MIBC stratification. The research focuses on the MIBC cohort from The Cancer Genome Atlas (TCGA) and the non-tumour dataset available from the Jack Birch Unit (JBU).

%%%%%% Cluster analysis %%%%%%
% The first chapter of the thesis employs aggressive gene filtering, retaining genes that are highly expressed in approximately 90\% of the TCGA samples, along with a simple clustering analysis.
A clustering analysis was performed on the gene expression data of the MIBC cohort from TCGA to establish the starting point for disease stratification in this project. Despite following a standard clustering approach, two of the five identified groups were Basal groups with different Interferon-$\gamma$ (IFN-$\gamma$) responses. This finding is consistent with the in-vitro work conducted at JBU, which, when combined with the clustering analysis, allows the Basal group to be further divided into three subgroups. The samples with the lowest IFN-$\gamma$ response also have the poorest survival rate. This demonstrates that even a standard clustering analysis can reveal new Basal groups within TCGA's MIBC cohort.


%%%%%% Network I %%%%%%
% Novelty: 
% 
% Selective edge pruning - finding 98 TFs which one of the lowest survival Basal group
The integrative co-expressed network approach is introduced in the second chapter to address the limitations of the current MIBC classifications. The novel method integrates the mutation burden by modelling the nodes' weights and the transcription factors (TFs) through selective edge pruning. The selective edge pruning reveals a subset of 98 TFs with an important role in both tissue differentiation and MIBC classification. The expression of the 98 TFs was used to stratify the MIBC, revealing a Basal group of 20 samples with one of the lowest survival rates found in the literature. The group exhibits squamous markers, has a low immune response, and some of the significantly expressed genes are responsible for the aggressiveness of the tumours, while others can be targeted to improve treatment response.

%%%%%% Network II %%%%%%
% Novelty
%
% highly connected communities and the high connected genes of 122
The network pipeline was further improved in the final chapter, where a co-expressed network was built from the expression data of non-tumour samples and subsequently used to stratify MIBC. The method demonstrated that some of the communities identified using the Stochastic Block Model (SBM) could be traced to the splitting of non-tumour datasets into new groups, with the potential to reveal new insights into bladder tissue differentiation. The new weight modifiers and the SBM revealed a group of communities with highly connected genes in the network and with various roles in tumours. These genes met the following conditions: high mutation burden and strong co-expression with other genes.



%%%%%% Conclusion %%%%%%
% The data integration occurred at several stages, the mutation burden at the weight's level where the TFs at the node's degree and expression of both the non-tumour and tumour both at the disease stratification.

Throughout the various integration methods and the network approach, different subsets of genes and new MIBC groups were found. This highlights that networks are a powerful bioinformatics tool for understanding cancer biology.

\newpage