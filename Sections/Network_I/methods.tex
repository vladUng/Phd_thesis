\subsection{Methods}


Start the section by explaining how PGCNA works and the role of the different parameters, covering the stages of PGCNA:
\begin{itemize}
    \item Gene selection
    \item Combination of multiple datasets
    \item Edges per node
    \item Correlation
    \item Community detection
    \item Module Community scores
    \item Module Evaluation Value
    \item Hierarchical clustering
    \item Visualising with Gephi
\end{itemize}

\subsubsection{Modified PGCNA}

Clearly separate the modifications I made to PGCNA and explain the reason behind them:

\begin{itemize}
    \item Already implemented:
          \begin{itemize}
              \item Allow more edges per node for the Transcription Factors (TF)
              \item Modify the pipeline to export the graph objects outside of the PGCNA developed by Mathew Care
              \item Add the modifiers to the genes to penalise/reward based on the mutation count.
          \end{itemize}
    \item On the list
          \begin{itemize}
              \item Change the community detection algorithm from Leiden to stochastic block model (SBM)(Zhang and Peixoto 2020). I read some criticism (blog, paper) of the current approaches in community detection and I want to explore the suggested alternative SBM.
          \end{itemize}
\end{itemize}


% \subsubsection{Building a network}

% \subsubsection{Analysing a network}

% \subsubsection{Finding communities}

% \subsubsection{Integrating data}

% \subsubsection{From gene communities back to samples subtypes}

% \subsubsection{Biological interpretation}