\section{Network I}


Key results:
\begin{itemize}
    \item Biological
          \begin{itemize}
              \item Integration of mutations on the healthy/tumour datasets
              \item PGCNA network with multiple datasets tumour, healthy
          \end{itemize}
    \item Engineering
          \begin{itemize}
              \item Pipeline developed to integrate mutations and the capability to easily change the ways on how the mutations are integrated.
              \item The tools created to analyse the networks. Mainly I'm referring to the community comparison.
          \end{itemize}
\end{itemize}



%%%%%%%%%%%%%%%%%%%%%%%%%%%%%%%%%%%%%%%%
\subsection{Overview}

This section needs to begin by addressing why a network approach, why the Gene expression is best represented by a network and the features that it enables us to build on that.

% 
The motivation behind using network/graph approaches is that they are closer to the biology of bladder cancer. More specifically, these allow us to represent pathways, and partial correlation between genes. Thus, a network approach is dealing both with the local and global effects of a gene. In addition, having links/connections between genes/nodes will enable to add weights (i.e importance) but also severe (in case of mutations that turns off a pathway).


\subsubsection{Datasets used}


The parameters that can be set through PGCNA are the following:
\begin{itemize}
    \item Dataset - P0, tum, AbsCa, AbsCa+P0
    \item Edges per gene - 3 by default
    \item Edges per TF - 50 by default
    \item Genes retention - how many of the most varied genes are retained. I run experiments for: 3K, 4K, 5K, 6K, 7K
    \item Modifiers - beta, norm3, standard
\end{itemize}


% This needs to be in the literature review side of things
% \import{Sections/Network_I/}{review.tex}


\import{Sections/Network_I/}{methods.tex}

\newpage

\subsection{Experiments Overview}


In the initial experiments two networks were constructed, one from the healthy dataset and the other from the tumour dataset (TCGA). Only P0 samples from the healthy datasets are initially used to build the networks as these represents an unaltered version of the bladder tissue. Compared with the AbsCa differentiated tissues dataset, P0 contains the noise in the in-situ tissue.

On these networks weight modifiers were applied, reward and penalising the genes that are mutated, proportionally to the mutations across the cohort. The hypothesis is that by applying modifiers it will improve the community separation inside the networks and each community will be representative of a biological process. Then, the most representative genes are selected for each community using ModCon score. Those  genes are used to stratify the MIBC samples by clustering their MEV.

\import{Sections/Network_I/}{tum.tex}
\newpage
\import{Sections/Network_I/}{p0.tex}
\newpage
\import{Sections/Network_I/}{h_derived.tex}
\newpage

\import{Sections/Network_I/}{gene_selection.tex}
\newpage

\newpage
\import{Sections/Network_I/}{selective_pruning.tex}

\newpage

\subsection{Discussion}


\subsection{Checkpoint}


Things left to do:
\begin{itemize}
    \item Experiments 
          \begin{todolist}
              \item Re-write the introduction to cover all the experiments, especially the two parts
          \end{todolist}
    \item P0 - experiments
          \begin{todolist}
              \item Put stats: How many of the genes selected w/ ModCon are TF. How does this compares to the initial ratio.
          \end{todolist}
    \item Tum - experiments
          \begin{todolist}
              \item To be completed
          \end{todolist}
    \item H derived experiments
          \begin{todolist}
              \item To be completed
          \end{todolist}
\end{itemize}
\vspace{1cm}
\textbf{Conceptual} regarding on network analysis. In essence, small questions that are trying to find the best network configuration:
\begin{todolist}
    \item Do we need more stats? How do the authors in PGCNA choose the number of edges per node? Can we apply it to our case?
    \item What defines a noisy network?
    \item How do we know if we lose some biological information?
\end{todolist}