\section{A network approach to subtype MIBC}


\vspace{3mm}
\fbox{
    \parbox{0.95\linewidth}{
      \begin{itemize}
        \item First attempt to integrate mutations and Transcription Factors
        \item Use non-cancerous data (P0) to inform MIBC subtyping 
        \item Develop a network pipeline (inspired from PGCNA) and tools to analyse results
      \end{itemize}
    }
}
\vspace{3mm}


%%%%%%%%%%%%%%%%%%%%%%%%%%%%%%%%%%%%%%%%
\subsection{Summary}

In the previous chapter, standard clustering methods and insights from in situ experiments (\citet{Baker2022-bj}) were utilised to stratify the muscle-invasive bladder cancer (MIBC) cohort from The Cancer Genome Atlas (TCGA). This work uncovered new MIBC subgroups with significant biological differences compared to those identified in the TCGA (\citet{Robertson2017-mg}) and consensus (\citet{Kamoun2020-tj}) stratification. This finding highlights the potential for discovering biologically relevant groups and underscores the need for new methods that combine data at the computational level.

Networks/Graphs methods have been previously used to stratify other cancers (glioblastoma - \citet{Care2019-ij}) and the gene expression data is represented in a way that enables further information integration.  Additionally, networks serve as powerful visualisation tools, enhancing the biological understanding of diseases.


This chapter delves into how mutations and transcription factors can be integrated into a network approach, aiming to uncover new MIBC subgroups. Two primary networks are constructed: one from the tumour dataset (Section \ref{s:N_I:tum}) and another from the healthy dataset (Section \ref{s:p0}). The experiments with the tumour network aim to demonstrate the potential of the network approach and to draw comparisons with the findings from the previous chapter. Conversely, the experiments with the healthy network investigate how non-cancerous information can inform MIBC stratification.

Furthermore, this chapter focuses on understanding how different parameters in the network pipeline influence the graphs and disease subtyping. It involves developing tools and methods to analyse the vast amount of results, highlighting the challenges inherent in working with unsupervised learning (in machine learning terminology), particularly in the context of highly heterogeneous data.

Overall, the work presented in this section demonstrates the potential of networks in MIBC subtyping. Although the initial methods for data integration show limited impact on disease subtyping, this chapter lays the groundwork for future chapters, where initial assumptions are revised, and new, adapted methods are employed.

\subsubsection{Datasets}

The MIBC cohort from The Cancer Genome Atlas \cite{Robertson2017-mg} and the P0 samples from Jack Birch Unit (JBU) are used. The first dataset is used in the first part of the chapter where a network is created from the genes expressed in the tumours and then subtype the cohort. While in second part, the network is constructed from the genes expressed in the P0 samples which are then used to subgroup the MIBC cohort from TCGA. Both datasets consists of Transcriptions Per Million (TPM) from RNA sequencing (RNAseq). 

The mutation information is coming from Whole Exon Sequencing (WES) of the MIBC cohort from TCGA. In this chapter only the mutation count, i.e. how much a gene is mutated across the cohort. In the following chapters the mutation type (e.g. if it is missense) is used to integrate more details.

The list of Transcription Factors (TFs) is taken from The Human Transcription Factors work \citet{Lambert2018-el}.



% This needs to be in the literature review side of things
% \import{Sections/Network_I/}{review.tex}


\import{Sections/Network_I/}{methods.tex}
\newpage
\import{Sections/Network_I/}{tum.tex}
\newpage
\import{Sections/Network_I/}{p0.tex}
\newpage
\import{Sections/Network_I/}{selective_pruning.tex}
\newpage

\subsection{Discussion}

This chapter presents the initial attempt of this project to apply a network-based approach for stratifying the muscle invasive bladder cancer (MIBC) cohort from the TCGA database. The approach involved integrating multiple data types at the computational stage, including gene expression from both cancerous and non-cancerous tissues, mutation burden, and transcription factors (TFs).

Two main co-expressed gene networks were explored: one constructed from tumour samples and the other from healthy tissue samples. The experiments demonstrated that integrating mutations and TFs was successful at the graph level, but had limited impact on MIBC subtyping. Network metrics varied between the modified networks (as shown in Figures \ref{fig:N_I:net_metrics_tum} and \ref{fig:N_I:net_metrics_p0}), and highly mutated genes were observed to change communities (Figure \ref{fig:N_I:p0_mut_burden}). The disease subgroups identified through this network approach showed different results compared to standard approaches, with only minor variations observed between networks modified by weight modifiers. This suggests that the weight modifier strategies, as depicted in Figure \ref{fig:N_I:modifiers}, require further refinement.

It was also noted that altering the number of edges per TF significantly impacted the Leiden algorithm's performance and the number of communities identified. Additionally, as highlighted by\citet{Peixoto2021-jx, Peixoto2023-rt}, models such as Leiden and Louvain are descriptive methods, and the communities they identify do not always reflect the underlying knowledge in the network. Consequently, an alternative community detection method will be explored in the following chapter.

Overall, the subsequent section will consider the following changes: 1) Analysing the effect of TFs in the networks to determine the most suitable value; 2) Utilising the entire healthy dataset to enhance gene representation in the tumour network; 3) Applying a different community detection algorithm; and 4) Improving the strategies for weight modifiers.

\newpage

\subsubsection{ToDo}

\begin{itemize}
    \item Introduction
    \begin{todolist}
        \item [\done] Why are we doing the experiments? - Because we want to study of the effect of the TF in the network
        \item [\done] What are the experiments performed?
        \item [\done] Community detection comparison
        \item [\done] Which dataset?
        \item [\done] How was the gene selection is done - v3 mostly
        \item [\done] Mention that some of the work done here was presented at the Complex Network conference 
    \end{todolist}
    \item General stats 
    \begin{todolist}
        \item [\done] Talk about the TF and how these are present in the dataset
        \item [\done] T In the cancerous and non-cancerous dataset
        \item [\done] Show some plots about their mutation count, mean expression
        \item [\done] T How many these are in the top 5000 genes
        \item Introduce the Human Transcription Factor list and why we want to prioritise them
    \end{todolist}
    \item Experiments 
    \begin{todolist}
        \item [\done] Using the tumour dataset as a starting point as the P0 wasn't that useful
        \item [\done] Using SBM for a change from Modularity Class - Comparing the methods
        \item [\done] Community size vs Modularity score
        \item [\done] Choosing control TFs 
        \item [\done] Range for 3-15 - Why choosing this range - all genes are selected by TF
    \end{todolist}
    \item Biological analysis 
    \begin{todolist}
        \item [\done] The DEA
        \item [\done] Morpheus - clustering
        \item [\done] Survival plot
        \item[\done] T Comparing CS models - can we use this rather then hierarchical clustering
        \item [\done] Comparing between subtypes        
        \item Significance in terms of the proteins, are those genes find somewhere
        \item [\done] Show the expression of some of the genes in Basal vs Luminal - bar plot
        \item [\done] Diff vs undiff
        \item [\done] Pi Plot for small basal
        \item [\done] TCGA metadata and the analysis
        \item [\done] GSEA enrichment on the Small Basal
        \item [\done] GSEA enrichment of the other 
    \end{todolist}
    \item Showing the advantages of the network
    \begin{todolist}
        \item [\done] Can we look at some of these genes that have neighbours? Are any of these useful
        \item Plotting the correlation distribution of these genes; before and after selective edge pruning
        \item [\done] Showing the neighbours for some of the TFs: BNC1, AHR, HES2
        \item Show the network for a standard network with Leiden and with SBM
    \end{todolist}
    \item Conclusion
    \begin{todolist}
        \item Integration of the TF works
        \item ModCon and MEV selection works 
    \end{todolist}
    \item Others 
    \begin{todolist}
        \item Check figures for being described properly, clear labelling, axis etc.
        \item Re-do the P0 graphs with the new plot
        \item Ensure that there is a logical narrative throughout the chapter
        \item Check for grammar, spelling
        \item Complement the Discussion
        \item Check if the introduction needs more addition
    \end{todolist}
\end{itemize}

\paragraph*{Things to do for paper}

\begin{itemize}
    \item Subtype exploration
    \begin{todolist}
        \item Further explore the unique pathways for the subtypes. A particular focus should be on the other subtypes: mes-like, lumInf (GU) and large luminal
        \item mes-like shares characteristics with the basal
        \item explore other values for ranking the most representative genes for a subtype
    \end{todolist}
    \item Advantages of the network
    \begin{todolist}
        \item Plotting the correlation distribution of these genes; before and after selective edge pruning
    \end{todolist}
\end{itemize}
