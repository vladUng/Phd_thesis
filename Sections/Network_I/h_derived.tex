\subsection{Healthy derived tumour stratification} \label{Sec:N_I:h_derived}

The hypothesis of this experiment is that by using the healthy communities the tunour dataset can be stratified differently. The experiments run here are:
\begin{itemize}
    \item All-healthy \- P0 + AbsCa + Undifferentiated
    \item TCGA dataset
    \item Leiden algorithm with Modularity Score
    \item 10TF
    \item 3EPG
    \item Modifiers standard/reward/penalisation
\end{itemize}

To test our hypothesis apart from the usual graphs used to analyse the networks (see below the list) there was some biological interepretation performed. This included running GO analysis and understanding which communities contributed to the subtypes.

This section will cover the usual graphs used to analyse the networks.
\begin{todolist}
    \item[\done] Leiden Rank compared to the modifiers
    \item[\done] Sankey plot with stratification with other methods: VU+in-situ + TCGA and consensus
    \item[\done] Community enrichment patterns
    \item Find significance between patterns.
    \item [\done] Survival plots.
    \item Survival tables with different checkpoints.
    \item Gene Ontology plots (?)
    \item What are the shared pathway? Are these scattered as in the P0 experiments?
    \item Show the genes that are not found in community 5. Exhibiting the need for remapping of all the healthy datasets.
    \item Show stats
    \begin{todolist}
        \item How many genes are shared between the tumour and healthy? All healty and all tumour? Selected healthy and all tumour / selected most varied?
        \item How many of the healthy genes are mutated?
    \end{todolist}
\end{todolist}

\subsubsection{Discussion}
From the analysis performed there are some clear findings, there are two patterns in community enrichment across samples and another one which is fuzzier. Apart from this the new subtypes stratify based on the healthy communitiy is different from the previous ones. \textbf{WHY so different? Not sure.} A possible explanation might be that healthy and tumour don't share that many genes. It's also worth mentioning that the reward networks seams to further separate the two main patterns in the community enrichment.

In this version of the network pipeline there are some parts that can be further improved.
\begin{enumerate}
    \item The Leiden community detection is flawed from the work from Tiago \cite{Peixoto2021-jx,Fortunato2016-tj}. Thus, we can use the Stochastic Block Model (SBM)\cite{Peixoto2017-ua}to update the network pipeline.
    \item The healthy datasets is not aligned with the latest gencode version. This was clear from the Community 5 which most of the genes are not found in the tumour dataset. Thus, the healthy datasets needs to remap.
    \item Gene selection. There is a need to either include more genes or select it differently to have a better representative of the mutated genes in the healthy.
    \item Modifiers are not powerful enough. There is a need to increase the resolution and maybe encode the oncogene and tumour supresor genes.
    \item Edge pruning. The current network approach draws inspiration from PGCNA which uses a very aggresive strategy, but we might lose some important information when we do that. SBM may be very sensitive to that and we may need to rethink the decision for TF. Also, an alternative is partial-correlation.
\end{enumerate}



