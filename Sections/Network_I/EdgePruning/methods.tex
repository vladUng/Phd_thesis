% Exp Setup
\section{Methods}

The experiments performed in this section are using the RNAseq gene expression from TCGA's MIBC cohort from TCGA and the transcription factor list was taken from \citet{Lambert2018-el}. As before, the Kallisto method was used to align the RNA-seq reads using genome version GRCh38 with Gencode annotation version 42. The gene selection strategy for the network is followed as in \cref{s:cs:methods} where genes are expressed in at least 90\% of the samples and and the top 5000 most varied genes by the median/standard deviation are kept. Out of the 5000 genes used for the network 325 are Human Transcription Factors based on \citet{Lambert2018-el}.

% Leiden and SBM configuration
The same network pipeline as described in \cref{fig:N_I:network_pipeline} was applied where at the community detection step both Leiden and DC-SBM are used and compared in the following sections. Leiden was run using Modularity Maximisation cost function and iterate over 10 times at each run. The degree-corrected Stochastic Block Model was used, with 700 iterations ($n\_iter=700$), the entire graph was sweeps up to 10 times ($mc\_iter = 10 sweeps$) and the distribution of the data was set to real exponential\footnote{As in the \href{https://graph-tool.skewed.de/static/doc/demos/inference/inference.html}{Graph-tool documentation} the distribution of the edge covariance can be specified given the bounds of the data, as seen in the referenced guide.} ($distribution = real\_exponential$). The number of iterations and swept insured that a detailed search for the communities is performed while keeping the computational times relatively low. The 'real-exponential' parameter it is used for edge weights that range from $[0, \infty]$ which is the case for the Spearman correlation values.

% Metrics
The performance of the Leiden algorithm is assessed by the Modularity Maximisation (described in \cref{s:lit:mod_max}), which measures the separation between the communities; higher values indicate better performance. SBM performance is measured by the Minimum Description Length (MDL), which measures the minimum information needed to describe the data; lower values indicate better performance. The number of communities found by the two community detection algorithms was also used as a heuristic to assess performance; it is generally thought that more communities indicate better performance \citep{Care2019-ij}. Unfortunately, at the time of running this experiment, there was no common metric available to compare the performance of the two methods. However, in the summer of 2023, \cite{Peixoto2023-mw}, Tiago Peixoto published research that adapted MDL to Modularity models (like Leiden).

%Choosing the Control TF
To account for the random effects of the TFs in the network, there are 10 control experiments with 325 non-TFs randomly selected and allowed a minimum degree from 3-15. The lower bound is given by the value used for the standard genes, while the higher limit was decided empirically as all the prioritised subsets are used for stratification (see \cref{fig:N_I:sel_tfs}); i.e. selected by ModCon based on their network importance. This set of experiments led to the identification of a subset of TFs with high biological significance, as discussed in \cref{s:N_I:sel_tfs}.


% Community detection
\subsection{Community detection} \label{s:N_I:methods_comm_detection}

Once the network is built, a community detection algorithm is applied to find the sub-networks of genes. The purpose is to identify genes that are correlated with each other, which may resemble some parts of the biological process. This hypothesis can be verified using Gene Set Enrichment Analysis (GSEA) or Gene Ontology (GO). It is worth noting that the ideal communities would be defined by biological pathways, but that information is often incomplete and not always available. Thus, the advantage of using a co-expressed network from mRNA gene expression is that it has the potential to be an affordable approximation.

In this project two main classes of community detection algorithms are explored: Leiden and Stochastic Block Model (SBM). The first algorithm is a popular method used to find partitions, blocks (=communities) used in the work of \citet{Care2019-ij}. However, Leiden may find non-existent structure in the data as explained in \citet{Peixoto2021-jx}. The SBM introduced by \citet{Peixoto2019-fg} addresses this issues by employing a Bayesian approach to find the blocks in the networks; both algorithms are covered in more depth in \cref{s:lit:comm_detect}.

In this project both methods are used. Initially, Leiden with Modularity Score is used in the Tumour (see \cref{s:N_I:tum}) and in P0 networks (see \cref{s:p0}). This is then followed by the experiments performed in the next chapter with all the non-tumour samples and the improved network pipeline in the last results chapter \cref{s:N_II}.
