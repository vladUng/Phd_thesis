\chapter{Selective edge pruning uncovers subset of regulatory genes} \label{s:N_I:sel_pruning}
\chaptermark{Selective edge pruning}
\vspace{3mm}
\fbox{
    \parbox{0.95\linewidth}{
      \begin{itemize}
        \item Finding the minimum degree for Transcription factors
        \item A subset of 98 genes TF are highly connected in the network
        \item Basal subset with the lowest survival rate
      \end{itemize}
    }
}
\vspace{3mm}

\section{Introduction}

% The aim to integrate the TF
The previous section has shown that the weight modifiers and selective edge pruning affect the graphs, community detection algorithms and the MIBC subtype. This part of the project aims to find the appropriate configuration for selective edge pruning and to understand their effects on community detection methods. Selective edge pruning reduces the number of edges in a network and prioritises a subset of genes, in this case \acrfull{tf}, to have more connections.

The objective of prioritising the \acrshort{tf} over the standard genes as these regulate the expression of the other genes, leading to more substantial biological influence by allowing more connections for the TF, the minimum degree of the node increases as well as their role in the network which is analogous with biology. 

% The aim is to find better community detection
An objective adopted for this section therefore is to compare the Leiden \citep{Traag2019-ne} community detection used in PGCNA with the degree-corrected Stochastic Block Model (DC-SBM) from \citep{Karrer2011-si, Peixoto2017-gc}. As covered in \cref{s:lit:comm_detect} from Background, Leiden is faster and more popular, but it is prone to finding patterns in noise, whereas the DC-SBM is slower but is unlikely to find non-existent communities. The differences between the two methods are covered in more detail in \cref{s:lit:comm_detect}.


The chapter begins by analysing the differences between the SBM and Leiden algorithms and how they respond to various configurations of selective edge pruning. During this comparison, a subset of 98 TF is identified as consistently highly connected, emerging even in the random-gene-selection control experiments. This subset of genes is further analysed and explored in \cref{s:N_I:sel_tfs} within the context of both tumour and non-tumour datasets. The analysis is then expanded in \cref{s:N_I:sel_tfs_bio}, where \acrfull{dea} is applied to the MIBC subtypes derived using the expression of these 98 genes.

\section{Aims}

This part of the project aims to:
\begin{itemize}
    \item Find the minimum degree for the TF nodes
    \item Understand the biological role of the 98 TF which are highly connected
    \item Investigate the subgroups derived from the expression of the 98 genes
    \item Comparing Leiden to the Stochastic Block Model
\end{itemize}


\import{Sections/Network_I/EdgePruning}{methods.tex}

\import{Sections/Network_I/EdgePruning}{com_detection_comp.tex}

\import{Sections/Network_I/EdgePruning}{results.tex}

\import{Sections/Network_I/EdgePruning}{dea.tex}


% Summary
\subsection{Summary}

% Recapitulate the aims and what have been done
This subsection began with two aims: identifying the appropriate community detection method and exploring the integration of Transcription Factors (TF) into the network. The comparison at the start of the section (\ref{s:N_I:sel_tf_com_det}) demonstrated that, despite the high computational cost of Stochastic Block Models (SBM), they offer more advantages over Modularity Maximisation algorithms such as Leiden. From a network perspective, the selective edge pruning method to integrate TF is effective, as those nodes exhibited a higher degree. Therefore, the two aims were achieved.

% Significance in cancer
Throughout the selective edge experiments, it was remarkable to find that by using controls and prioritising non-TF, a subset of 'naturally' well-connected TFs in the healthy dataset could be derived with important biological roles in tumour biology. The analysis in this section indicates that the subgroup of 98 TF has essential biological importance. Hierarchical clustering was applied to this subset of genes, yielding subtypes with significantly different survival outcomes (\cref{fig:N_I:sel_tfs_cs_analysis}). Of the five newly found groups, basal 5 has the lowest survival with squamous markers, and basal 3 resembles the mesenchymal group from the Lund cohort.

While the subset of TF was informative for the Basal subgroup, it was less successful for the luminal subgroup. The previously derived luminal subtypes (consensus, TCGA, Lund, and the work in the previous chapter) are grouped into one large group. Looking at a known luminal marker, \textit{FGFR3}, it was observed that this gene has a low degree in the network. In addition, some other known markers were left out in the gene selection process (e.g., \textit{PPARG}). This may also be explained by the fact that luminal tumours are "molecular closer" to differentiated bladder tissue compared to the basal subtypes, which are undifferentiated urothelium. The high variance introduced by the basal subtype might explain why other known gene markers were missed.

\section{Discussion}

% Main results
Selective edge pruning has proven to be more than just a technique for integrating additional data into the network; it is also a novel method for identifying highly correlated genes in gene expression data. By combining experiments with controls (random genes), selective edge pruning, along with ModCon, successfully extracted 98 TF (out of the 325 in the 5000-node network), which were then further analysed. The study of this subset of genes revealed five distinct MIBC groups, including three basal subtypes, a large luminal group, and a luminal-infiltrated group.

% Gene expression
The analysis of the 98 TF identified a basal group (Basal 5) consisting of 20 patients with less than a 20\% chance of survival after 1.7 years (20 months), representing one of the poorest prognoses observed in the project. This group exhibited squamous markers, had a low immune response, and 7 out of the 10 most significantly expressed genes had previously been studied in cancer. Some of these 7 genes are associated with the aggressiveness of the tumours, while others could be targeted to improve treatment response.

% Other groups
Another subgroup, Basal 3, consisting of 30 samples, was identified, which included tumours previously classified as Mes-like by the Lund classifier \citep{Marzouka2018-ge}. Both Basal 3 and Basal 4 exhibit similar survival prognoses, although Basal 3 has a more favourable outlook than Basal 4; see \cref{fig:N_I:sel_tfs_survival}. The significance of identifying a subtype similar to the Mes-like group from the Lund classifier lies in the fact that only the gene expression of the 98 TF was used, without the \acrlong{ihc} data employed in the Lund cohort. This underscores the critical influence of the TF subset on bladder tumours.



% New markers
\subsubsection*{Potential new markers}

% the two lists of basal/luminal
Using the analysis performed on the non-tumour dataset and the gene expression data from TCGA, the 98 TF genes were reduced to two different lists of markers for basal and luminal subtypes (\cref{tab:N_I:genes_lum_basal}). Many genes from these two lists were validated by other research work, which strengthened the selective edge pruning as a method for extracting highly connected genes.

% the list for non-tumour data
Extending the findings beyond cancer, some of the 98 TF genes were classified by their expression in the non-tumour dataset. The analysis performed (\cref{s:N:sel_tf_diff_status}) proposes three lists of genes specific to P0, UD, and ABS-Ca, as well as another three between these categories. This refined list of genes may help biologists better understand urothelium differentiation, which in turn will also aid in elucidating the basal/luminal molecular differences.

% Further work
Considering the limitations, the analysis performed suggests that both the subset of 98 TF and the subtypes need more in-depth exploration, even selecting a few targeted genes to study the bladder response \textit{in vitro}. It also suggests that the selective edge pruning is working and exhibits highly relevant biological results.

% Comm detection
\subsubsection*{Community detection}

Varying the number of edges per TF significantly impacts the community detection algorithms, Leiden more than SBM. It was observed that the former finds fewer communities as the nodes' degree is increased, while SBM is more stable. This, along with the work from \citet{Peixoto2023-se, Peixoto2023-rt}, which shows that Modularity Maximisation methods are prone to finding patterns in noise, made SBM the preferred method in this project. In this chapter, the simple degree-corrected version of SBM was used, but a hierarchical version of it will be used in the next chapter. The hierarchical SBM is more complex and computationally expensive than the standard version, but it does not have a threshold on the number of communities it can find; limit discussed in \cref{s:lit:sbm}.


\subsection* {Limitations}

One of the limitations of the work done in selective edge pruning was that only ten control networks were used for comparison. This was due to the high computational time (a few hours) required to run the SBM, the lengthy process of loading the network output\footnote{During the network runs, and SBM, a lot of metadata was saved, which helped to debug, understand, and analyse the networks. This made the network output reasonably large, taking a considerable RAM footprint. For example, analysing the results for the 10 controls and experiments takes approximately 40GB of RAM.} and the limited time in a PhD project. Another possible limitation is using only 5000 genes for a network using SBM. As it will be seen in the next chapter \cref{s:N_II}, the stochastic model can find smaller communities in a large network. This means that more genes can be included, which will make the network very dense, so extra rules for edge pruning might be needed. However, from a biological perspective, re-running the selective edge pruning on larger networks and more controls may further enforce that a subset of TF o-expressed with many genes in bladder cancer.


This and the previous chapter explored three types of data integration in a graph: at a network level, edge weights, and the number of connections. The first two strategies need further refining, while for the selective edge pruning, a minimum degree of 6 for TF and 3 for standard genes was found to be appropriate. The subsequent section will consider the following changes:  1) utilising the entire non-tumour dataset to enhance gene representation in the tumour network; 2) using a hierarchical stochastic block model, and 3) improving the strategies for weight modifiers.


\subsection{Chapter contributions}

The research covered in this chapter was presented at the \textit{\href{https://2023.complexnetworks.org/}{Complex Network Conference}} from Menton, France, in 2023.

\begin{itemize}
    \item Selective edge pruning as a method to find highly correlated genes in gene expression datasets 
    \item 98 TF drive a three-way basal split
    \item The Basal group of 20 samples exhibiting squamous markers has the lowest survival prognosis seen in the literature
    \item Compares two classes of community detection methods in the biological networks and finds SBM to perform better 
\end{itemize}

\subsection{Code availability}

The experiments performed in this section use a modified version of the PGCNA Python package developed by \citep{Care2019-ij}, in \textbf{pgcna3.py}. The code for analysing the experiment and control networks is in the \textbf{selective\_edge\_pruning} notebook, along with the finding of the subset of the 98 TFs. Their biological analysis is covered in \textbf{selectred\_ctrl\_tf} notebook. See \cref{tab:ap:software} from Appendix to the corresponding notebook and the links to their location in the project's \href{https://github.com/vladUng/PhD_thesis_exp/}{GitHub repository}.


