%%%%%%%%%%%% Selective edge pruning %%%%%%%%%%%%
\chapter{Selective edge pruning} \label{s:ap:sel_prun}

% Leiden vs SBM
\section{Leiden vs SBM} \label{s:ap:leiden_sbm}

The below figures are supporting the work presented in \cref{s:N_I:sel_tf_com_det} where the two set of traces, one for Leiden (\cref{fig:ap:sbm_com_size}) and one for SBM (\cref{fig:ap:sbm_com_size}), are shown in a single figure in \cref{fig:N_I:comp_size_com_det}. 

The advantage of showing the two figures separately is that the variance in the community size is illustrated clearly as well as the difference in the number of communities found.

% Comunity sizes
\begin{figure}[!h]
    \centering
    \begin{subfigure}[!t]{1.0\textwidth}
        \includegraphics[width=\textwidth]{Sections/Network_I/Resources/selective_pruning/com_comp/sbm_comNum_sel_prun.png}
        \caption{SBM}
        \label{fig:ap:sbm_com_size}
    \end{subfigure} 
    \begin{subfigure}[!t]{1.0\textwidth}
        \includegraphics[width=\linewidth]{Sections/Network_I/Resources/selective_pruning/leid_comNum_sel_prun.png}
        \caption{Leiden}
        \label{fig:ap:leid_com_size}
    \end{subfigure}\hspace{\fill}  
    \caption{Community size comparison between Leiden and SBM. This serves as a supporting material to the work done in \cref{s:N_I:sel_pruning}. It shows that SBM tends to find more communities compared to Leiden.}
    \label{fig:ap:com_size_comp}
\end{figure}

% Cluster analysis
\section{Cluster analysis} \label{s:ap:sel_tfs_cs}

The clustering analysis below supports the MIBC groups found using the expression of the 98 TFs genes which are highly correlated with other genes in the tumour network, see \cref{s:N_I:sel_tfs_mibc}.

% Talk about the clustering
Hierarchical clustering was applied to the tumour expression of the 98 TFs after 11 samples previously identified as outliers were removed\footnote{In this context, outliers are samples that cluster in 1-3 groups, making it hard to interpret the dendrogram from the hierarchical clustering}, leaving 392 samples for clustering. The tumour\footnote{TCGA's MIBC cohort} data was log2(TPM+1) transformed, normalised by the quantiles, and then agglomerative clustering with average linkage was applied on the 1-Pearson correlation distance\footnote{In this case, for both clustering and visualisation, Morpheus from the Broad Institute was used \cite{Broad-InstituteUnknown-kn}}. This method was preferred over those developed in the previous chapter because the number of genes was small and the heatmap from Morpheus is a useful tool for showing the gene expression specific to a group. The heatmap and the output are shown in \cref{fig:ap:morph_sel_tfs} in the Appendix.

% describe morpheus
A dendrogram cut of 15 was chosen as it split both the Luminal and Basal Groups and there are relatively small groups. From the heatmap \cref{fig:ap:morph_sel_tfs} it can be noticed that there is a large group of Luminal samples and a smaller one. The Basal group is split into three subgroups, a large one, a medium size which groups most of the Mes-like tumours from Lund classifier and a smaller ones which has a lower Infiltration/Stromal/Estimate scores compared to the other 2 Basal groups. Apart from these 2 groups, there are outliers samples that are grouped in 1 or 2 clusters, suggesting that they have particular molecular profiles.

The groups smaller than 1\% of the cohort size (equivalent to 4 samples) are removed, resulting in a dataset of 378 samples grouped into 5 major groups. 


%  Morpheus hierarchical clustering
% \begin{figure}[!htb]   
\begin{sidewaysfigure}
    \centering
    \includegraphics[width=1.0\textwidth,keepaspectratio]{Sections/Network_I/Resources/selective_pruning/15_CS_norm_sel_tfs.png}
      \caption{Hierarchical clustering of the 98 TFs found in \cref{s:N_I:sel_tfs}. The columns at the top of the heatmap represents the previous classification (TCGA \cite{Robertson2017-mg}, Consensus \cite{Kamoun2020-tj}, Lund \cite{Marzouka2018-ge} and the stratification from \cref{s:clustering_analysis}) as well as the Immune, Stromal and ESTIMATE scores available with the TCGA cohort.}
    \label{fig:ap:morph_sel_tfs}
\end{sidewaysfigure}

% tumour vs non-tumour
\section{Tumour vs Non-tumour gene expression} \label{s:ap:tum_vs_non-tumour}


The scatter plot complements the work in \cref{s:N_I:sel_tfs_bio} which explores the 98 TFs in both tumour and non-tumour datasets.

\Cref{fig:ap:sel_tfs_mean} depicts the mean expression of the 98 TFs genes found in the previous subsection. The log plot shows the non-cancerous mean on the y-axis and the tumour mean expression on the x-axis, the size and colour of the points is proportional to the mutation burden across the MIBC cohort from TCGA. The genes higher on the y-axis have a higher expression in the non-cancerous, similarly for x-axis, further on the right hand side, higher average value in the tumour cohort. \textit{ELF3} it is on the top right corner meaning that it is expressed both in the tumour and non-cancerous datasets, and it is also highly mutated in MIBC. \textit{BNC1} is on he left corner, having lower expression across the samples.

% Tum vs non-cancerous dataset.
\begin{figure}[!htb]
    \centering
    \includegraphics[width=1.0\textwidth,keepaspectratio]{Sections/Network_I/Resources/selective_pruning/sel_tfs/sel_tfs_mean_tum_healthy.png}
    \caption{Selected TFs expression in the MIBC TCGA cohort and in the non-cancerous. Both the colour and size of the points are proportional to the mutation burden in the TCGA cohort.}
    \label{fig:ap:sel_tfs_mean}
\end{figure}



% Metadata of the hierarchical clustering
\section{TCGA metadata and the clusters based on the 98 TFs} \label{s:ap:sel_prun_tcga_meta}


\begin{figure}[!htb]   
    \centering
    \includegraphics[width=1.0\textwidth,keepaspectratio]{Sections/Network_I/Resources/selective_pruning/sel_tfs/sel_tfs_tcga_meta.png}
      \caption{The metadata from TCGA \cite{Robertson2017-mg} across the subtypes derived from applying hierarchical clustering on the expression of the 98 TFs from \cref{fig:N_I:sel_tfs}. }
    \label{fig:ap:sel_tfs_tcga_metadata}
\end{figure}

\begin{figure}[!htb]   
    \centering
    \includegraphics[width=1.0\textwidth,keepaspectratio]{Sections/Network_I/Resources/selective_pruning/sel_tfs/sel_tfs_apobec_meta.png}
      \caption{Heatmap of the the APOBEC mutations in TCGA.}
    \label{fig:ap:sel_tfs_tcga_meta_apobec}
\end{figure}

\newpage 

\section{Basal and Luminal markers on consensus subgroups} \label{s:ap:sel_prun_markers}

% The Basal/Luminal Markers in the consensus
\begin{figure}[H]
    \captionsetup[subfigure]{justification=Centering}
  % consensus
  \begin{subfigure}[!t]{1.0\textwidth}
    \includegraphics[width=1.0\textwidth,height=1.0\textheight,keepaspectratio]{Sections/Network_I/Resources/selective_pruning/log2_consensus_basal.png}
    \caption{Basal}
    \label{fig:ap:box_basal_consensus}
    \end{subfigure}
  \begin{subfigure}[!t]{1.0\textwidth}
      \includegraphics[width=1.0\textwidth,height=1.0\textheight,keepaspectratio]{Sections/Network_I/Resources/selective_pruning/log2_consensus_lum.png}
      \caption{Luminal}
      \label{fig:ap:box_luminal_consensus}
  \end{subfigure}
  \caption{Box plots showing the $log2(TPM+1)$ of the \textbf{luminal markers} over the consensus and groups derived from the hierarchical clustering applied in this chapter.}
  \label{fig:ap:box_consensus}
\end{figure}



% Pi-plots for GSEA
\section{Pi-plots for GSEA} \label{s:ap:sel_prun_pi}

This section contains the Pi-plots for each of the MIBC subtypes derived from the gene expression of the 98 TFs from \cref{s:N_I:sel_tfs}. The method on how the DEA was performed and the Pi-plots generated can be found in \cref{s:lit:dea,s:lit:pi,s:lit:gsea}.


\begin{figure}[H]
    \centering
    \begin{subfigure}[!t]{1.0\linewidth}
        \includegraphics[width=\textwidth,keepaspectratio]{Sections/Network_I/Resources/selective_pruning/pi_gsea/pi_largeBasal.png}
        \caption{Large Basal}
        \label{fig:ap:pi_basal}
    \end{subfigure}
    \begin{subfigure}[!t]{1.0\textwidth}
        \includegraphics[width=\textwidth,keepaspectratio]{Sections/Network_I/Resources/selective_pruning/pi_gsea/pi_largeLuminal.png}
        \caption{Large luminal}
        \label{fig:ap:pi_lum}
    \end{subfigure}
    \caption{Pi plots for mes-like and lumInf from \cref{s:N_I:sel_tfs_subtypes}}
    \label{fig:ap:pi_other_values_I}
\end{figure}

\begin{figure}[!h]
    \centering
    \begin{subfigure}[!t]{1.0\textwidth}
        \includegraphics[width=\textwidth,keepaspectratio]{Sections/Network_I/Resources/selective_pruning/pi_gsea/pi_mesLike.png}
        \caption{Mes-like}
        \label{fig:ap:mes_like}
    \end{subfigure}
    \begin{subfigure}[!t]{1.0\textwidth}
        \includegraphics[width=\textwidth,keepaspectratio]{Sections/Network_I/Resources/selective_pruning/pi_gsea/pi_lumInf.png}
        \caption{Luminal infiltrated}
        \label{fig:ap:lumInf}
    \end{subfigure} 
    \caption{Pi plots for mes-like and lumInf from \cref{s:N_I:sel_tfs_subtypes}}
    \label{fig:ap:pi_other_values_II}
\end{figure}

\newpage

% GSEA - hallmarks
\section{GSEA output (Hallmarks)} \label{s:ap:hallmarks}

\begin{table}[H]
  \centering
  \scriptsize
  \begin{tabularx}{\textwidth}{>{\hsize=1.5\hsize}X|>{\hsize=0.4\hsize}X|>{\hsize=0.4\hsize}X|>{\hsize=0.6\hsize}X|>{\hsize=0.4\hsize}X|>{\hsize=0.4\hsize}X}
    \toprule
    \textbf{Term} & \textbf{NES} & \textbf{FDR q-val} & \textbf{\# lead} & \textbf{\# matched} & \textbf{ratio} \\
    \midrule
    \multicolumn{6}{c}{\textbf{smallBasal}} \\
    \midrule
    MYC TARGETS V1 & 1.909 & 0 & 149 & 40 & 0.268 \\
    \midrule
    MITOTIC SPINDLE & 1.887 & 0 & 138 & 61 & 0.442 \\
    \midrule
    TGF BETA SIGNALING & 1.863 & 0 & 28 & 15 & 0.536 \\
    \midrule
    \multicolumn{6}{c}{\textbf{largeBasal}} \\
    \midrule
    KRAS SIGNALING UP & 2.384 & 0 & 132 & 104 & 0.788 \\
    \midrule
    \multicolumn{6}{c}{\textbf{lumInf}} \\
    \midrule
    CHOLESTEROL HOMEOSTASIS & 1.892 & 0 & 33 & 20 & 0.606 \\
    \midrule
    APOPTOSIS & 1.733 & 0 & 61 & 37 & 0.607 \\
    \midrule
    \multicolumn{6}{c}{\textbf{largeLuminal}} \\
    \midrule
    DNA REPAIR & 1.617 & 0.004 & 77 & 12 & 0.156 \\
    \midrule
    PEROXISOME & 1.608 & 0.003 & 57 & 22 & 0.386 \\
    \midrule
    FATTY ACID METABOLISM & 1.552 & 0.004 & 71 & 38 & 0.535 \\
    \midrule
    PROTEIN SECRETION & 1.549 & 0.003 & 42 & 11 & 0.262 \\
    \midrule
    BILE ACID METABOLISM & 1.46 & 0.008 & 59 & 19 & 0.322 \\
    \bottomrule
  \end{tabularx}
  \caption{Normalised Enrichment Score (NES), False Discovery Rate (FDR) q-val, and lead gene statistics for different subtypes. The lead genes from a pathway are selected by GSEAY based on when the NES reached its peak.}
  \label{ap:tab:gsea_hallmark}
\end{table}

\newpage

% % GSEA - oncoSig
% \section{GSEA output (OncoSig)} \label{s:ap:sel_prun_oncosig}

% % Table for OnCoSig
% \begin{table}[H]
%   \centering
%   \scriptsize
%   \begin{tabularx}{\textwidth}{>{\hsize=1.5\hsize}X|>{\hsize=0.4\hsize}X|>{\hsize=0.4\hsize}X|>{\hsize=0.6\hsize}X|>{\hsize=0.4\hsize}X|>{\hsize=0.4\hsize}X}
%     \toprule
%     \textbf{Term} & \textbf{NES} & \textbf{FDR q-val} & \textbf{\# lead genes} & \textbf{\# matchedl} & \textbf{ratio matched} \\
%     \midrule
%     \multicolumn{6}{c}{\textbf{smallBasal}} \\
%     \midrule
%     SINGH KRAS DEPENDENCY SIGNATURE & 2.121 & 0 & 17 & 10 & 0.588 \\
%     \midrule
%     TBK1.DF DN & 2.105 & 0 & 206 & 132 & 0.641 \\
%     \midrule
%     EIF4E DN & 2.084 & 0 & 53 & 44 & 0.83 \\
%     \midrule
%     PGF UP.V1 UP & 2.002 & 0 & 111 & 67 & 0.604 \\
%     \midrule
%     ERBB2 UP.V1 DN & 1.877 & 0 & 110 & 67 & 0.609 \\
%     \midrule
%     GCNP SHH UP LATE.V1 UP & 1.863 & 0 & 120 & 52 & 0.433 \\
%     \midrule
%     P53 DN.V1 UP & 1.862 & 0 & 68 & 65 & 0.956 \\
%     \midrule
%     RB P130 DN.V1 DN & 1.862 & 0 & 82 & 52 & 0.634 \\
%     \midrule
%     \multicolumn{6}{c}{\textbf{largeBasal}} \\
%     \midrule
%     CSR LATE UP.V1 UP & 2.382 & 0 & 115 & 86 & 0.748 \\
%     \midrule
%     TBK1.DF UP & 2.332 & 0 & 173 & 135 & 0.78 \\
%     \midrule
%     CSR EARLY UP.V1 UP & 2.326 & 0 & 110 & 74 & 0.673 \\
%     \midrule
%     \multicolumn{6}{c}{\textbf{mesLike}} \\
%     \midrule
%     CORDENONSI YAP CONSERVED SIGNATURE & 2.49 & 0 & 48 & 39 & 0.812 \\
%     \midrule
%     LEF1 UP.V1 UP & 2.423 & 0 & 125 & 110 & 0.88 \\
%     \midrule
%     RB P107 DN.V1 UP & 2.316 & 0 & 84 & 71 & 0.845 \\
%     \midrule
%     LTE2 UP.V1 DN & 2.265 & 0 & 118 & 86 & 0.729 \\
%     \midrule
%     \multicolumn{6}{c}{\textbf{lumInf}} \\
%     \midrule
%     BCAT.100 UP.V1 UP & 2.067 & 0 & 24 & 22 & 0.917 \\
%     \midrule
%     CSR LATE UP.V1 DN & 1.964 & 0 & 70 & 49 & 0.7 \\
%     \midrule
%     AKT UP.V1 DN & 1.889 & 0 & 90 & 66 & 0.733 \\
%     \midrule
%     ESC J1 UP LATE.V1 UP & 1.828 & 0 & 81 & 67 & 0.827 \\
%     \midrule
%     \multicolumn{6}{c}{\textbf{largeLuminal}} \\
%     CSR EARLY UP.V1 DN & 1.679 & 0.002 & 73 & 32 & 0.438 \\
%     \midrule
%     MYC UP.V1 DN & 1.613 & 0.002 & 77 & 38 & 0.494 \\
%     \bottomrule
%   \end{tabularx}
%    \caption{Normalised Enrichment Score (NES), False Discovery Rate (FDR) q-val, and lead gene statistics for different subtypes and terms in bladder cancer biology from the OncoSig database. The lead genes from a pathway are selected by GSEAPY based on when the NES reached its peak.}
%   \label{ap:tab:gsea_oncosig}
% \end{table}

% \newpage



% GSEA plots
\section{GSEA plots top 10 by Enrichment score}



\begin{figure}[!htb]
    \centering
    \includegraphics[width=\textwidth,keepaspectratio]{Sections/Network_I/Resources/selective_pruning/gsea/smallBasal_10_top_manTerms.png}
    \caption{Small Basal}
    \label{fig:ap:gsea_smallBasal}
\end{figure}


\begin{figure}[!htb]
    \centering
    \includegraphics[width=\textwidth,keepaspectratio]{Sections/Network_I/Resources/selective_pruning/gsea/largeBasal_10_top_manTerms.png}
    \caption{GSEA output of the Basal group sfor the groups derived using Selective Edge pruning in  \cref{s:N_I:sel_tfs_subtypes}}
    \label{fig:ap:gsea_largeBasal}
\end{figure}

\begin{figure}[!htb]
    \centering
    \includegraphics[width=\textwidth,keepaspectratio]{Sections/Network_I/Resources/selective_pruning/gsea/lumInf_10_top_manTerms.png}
    \caption{GSEA output of the Luminal Infiltrated group for the groups derived using Selective Edge pruning in  \cref{s:N_I:sel_tfs_subtypes}}
    \label{fig:ap:gsea_lumInf}
\end{figure}

\begin{figure}[!htb]
    \centering
    \includegraphics[width=\textwidth,keepaspectratio]{Sections/Network_I/Resources/selective_pruning/gsea/mesLike_10_top_manTerms.png}
    \caption{GSEA output of the Mes-like group for the groups derived using Selective Edge pruning in  \cref{s:N_I:sel_tfs_subtypes}}
    \label{fig:ap:gsea_mesLike}
\end{figure}

\begin{figure}[!htb]
    \centering
    \includegraphics[width=\textwidth,keepaspectratio]{Sections/Network_I/Resources/selective_pruning/gsea/largeLuminal_10_top_manTerms.png}
    \caption{GSEA output of the Luminal group for the groups derived using Selective Edge pruning in  \cref{s:N_I:sel_tfs_subtypes}}
    \label{fig:ap:gsea_largeLuminal}
\end{figure}


