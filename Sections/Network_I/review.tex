
\subsection{Network approaches}

So far in my literature review of the graphs/network approaches to biological networks, I found out that the processing steps are common across the solutions propsosed. Overall, the workflow is:
\begin{itemize}
    \item Reducing the data - This comes in different shapes and forms but all have the same goal, create a smaller adjacent matrix from which a network can then be created.
    \begin{itemize}
        \item In PCGNA - the median of the spearman correlation is used to reduce all the correlation across the datasets.
        \item In WCGNA - another correlation
        \item The Graph NN - embedding space
        \item The encoder - another embedding space
    \end{itemize}
    \item Do some pruning on the the edges
    \item The most aggressive being PGCNA 
    \item Apply some clustering to find the communities or the genes correlated 
\end{itemize}


The problem with the embedding space is that it may be less “open” then the spearman correlation.


\subsubsection{Previous work}

In this work the authors \cite{Sobolevsky2022-di} compare Louivan \cite{} and Leiden to Grap Neural Networks.

\subsubsection{Describing a network}

The motivation behind using network/graph approaches is that they are closer to the biology of bladder cancer. More specifically, these allow us to represent pathways, and partial correlation between genes. Thus, a network approach is dealing both with the local and global effects of a gene. In addition, having links/connections between genes/nodes will enable to add weights (i.e importance) but also severe (in case of mutations that turns off a pathway).
\\~\\
The network approach can be split into several sub-parts:
\begin{enumerate}
    \item \textbf{Building the network/graph}
    \begin{itemize}
        \item This covers the work on CoRegNet \citet{Nicolle2015-tn} and PGCNA2 \citet{Care2019-ij}
    \end{itemize}
    \item \textbf{Finding the key genes in the network}. This problem comes in different faces: the problem of node influence, node centrality, network's hub etc. Some papers that look at this are:
    \begin{itemize}
        \item Integrated Value of Influence (IVI) \citet{Salavaty2020-wo} - This is a very good paper which combines different metrics to find the node of influence in a biological network. It justifies why choosing 6 popular metrics and how are integrated. I think this should be ok to use it.  - \textit{See more notes at the end}
        \item Hub Discovery in partial Correlation Graphs by \citet{Hero2012-ch} - Seams good but it's very mathematical and couldn't find any implementation.
        \item (+) other methods like Autoencoders, Graph Neural Networks, and Cartesian Genetic Programming? (the dataset might be too big though)
    \end{itemize}
    \item \textbf{Propagating the data to network}. This refers to how to integrate other datasets 
    \begin{itemize}
        \item There is some relevant work from \citet{He2017-dj} where they have integrated somatic mutations with gene expression. See below for a detailed description.
    \end{itemize}
    \item \textbf{How the data is visualised}
    \begin{itemize}
        \item In both the CoRegNet and PGCNA2 the authors have some recommendations on what tools to use to visualise the network. At the moment, I haven't revised these but my specs are: interactive, being able to filter/select different nodes.
    \end{itemize}
\end{enumerate}


The work from \citet{He2017-dj} is \textbf{very relevant to this stage} as they have integrated somatic mutations with gene expression in a network approach.  They employed a very similar method to what we were thinking to do:
\begin{itemize}
    \item Have two matrixes, one for mutations and one for gene expression
    \item Created the Gene expression network, using Spearman correlation. They filtered out the genes that have a poor correlation
    \item Propagate the mutations to the network using diffusion (?)
    \item Group together the matrix with both gene expression and mutations
    \item Repeat 100 times and then did some clustering

\end{itemize}

The last two steps are very similar to the work done by \citet{Robertson2017-mg}. In terms of the interpretation, \citet{He2017-dj} looked at the clusters from the last step and analyse the differential mutations and gene expression.
\\~\\
It's worth mentioning that this work is building on \citet{Hofree2013-ld} which introduced a network-based stratification (?) method based on mutations. Nevertheless, they hinted at using an integrative approach (maybe that's what sparked \citet{He2017-dj}). \citet{Hofree2013-ld} not only inspired \citet{He2017-dj} but also other network-based research such as:
\begin{itemize}
    \item \citet{Kong2022-gv} (very recent). Not very relevant to us, as they are looking at immune responses (?) no stratification.
     \item DeRegNet by \citet{Winkler2022-vg}. Essentially this is an improved version of Gene Set Enrichment Analysis (GSEA) by using networks and integrating multiple datasets.
     \item Breast cancer molecular subtype using Deep Clustering approach by \citet{Rohani2020-px}. The work seems to be only in mutations but it also refers \citet{Curtis2012-ff} in which iCluster was applied for breast cancer subtyping. \citet{Rohani2020-px} are doing more or less what we want to do but just with the mutations data. However, the workflow is very similar to what we were thinking:
     \begin{itemize}
         \item Build a network (they've used STRING for gene network interaction)
         \item Apply mutations to the network
         \item Dimension reduction with Autoencoders
         \item Clustering
         \item They've performed separated gene set enrichment analysis
     \end{itemize}
\end{itemize}

\textbf{Critics with \citet{He2017-dj}:}
\begin{itemize}
    \item I am not sure how white-box is their method and how easy is to identify the differential mutations
    \item They only look at the different survival rates, not biological validation
\end{itemize}

\textbf{Next:}
\begin{itemize}
    \item I've tried to find some follow-ups but I wasn't able to find that much.
\end{itemize}


\subsubsection{Integrated Value of Influence}

The central focus of the paper is to identify the most influential nodes in the network. This can mean the nodes that play an important role, have lots of connections w/ the other nodes. 

The paper is relevant to my work for several reasons:
\begin{itemize}
    \item it offers a good review of the metrics used to identify the most influential nodes
    \item  It justifies how the integration is done, but thought the explanation of addition/multiplication was a bit too much and unnecessary
\end{itemize}

--- 
Some ways to look at the influence of a node: 

\begin{itemize}
    \item They talk a lot about \textbf{centrality} as a way to measure influence.
    \item Mention of \textbf{betweenness} = "defined as the tendency of a node to be on the shortest path between nodes in a graph".
    \item \textbf{Collective influence} =  "is a novel global centrality metric that measures the collective number of nodes that can be reached from a given node" 
    \item\textbf{ Neighbourhood connectivity} = "is a semi-local centrality measure of a network that deals with the connectivity (number of neighbours) of nodes. "

\end{itemize}

Their novel part or what are the limitations that they are addressing through the Integrated Value of Influence (IVI):

\begin{enumerate}
    \item degree centrality 
    \item ClusterRank
    \item Neighbourhood connectivity - This is used to overcome the issue of positional bias of betweenness centrality
    \item local H index
    \item Betweenness centrality  
    \item Collective influence 
\end{enumerate}


They say that each of the metrics captures a different topological dim of the graph, including local, semi-local and global topology. Another advantage of their work is that: \textit{None of them requires a fully connected graph or module to be calculated}.

\subsubsection{Comparing networks}


\import{Sections/}{graph_nn.tex}

\newpage


 


