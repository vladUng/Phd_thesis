\subsection{Methods} \label{Sec:N_II:methods}

\paragraph*{iNet package}

\paragraph*{Creating the network}

\begin{itemize}
    \item Gene filtering
    \item Gene selection
    \item Selective edge pruning
    \item Weight modifiers
\end{itemize}

\paragraph*{Community Detection}

\begin{itemize}
    \item hSBM
    \item Leiden?
    \item standard SBM?
\end{itemize}











To test our hypothesis apart from the usual graphs used to analyse the networks (see below the list) there was some biological interpretation performed. This included running GO analysis and understanding which communities contributed to the subtypes.

This section will cover the usual graphs used to analyse the networks.
\begin{todolist}
    \item[\done] Leiden Rank compared to the modifiers
    \item[\done] Sankey plot with stratification with other methods: VU+in-situ + TCGA and consensus
    \item[\done] Community enrichment patterns
    \item [\done] Survival plots.
    \item Survival tables with different checkpoints.
    \item Show the genes that are not found in community 5. Exhibiting the need for remapping of all the healthy datasets.
    \item Gene stats stats
    \begin{todolist}
        \item How many genes are shared between the tumour and healthy? All healty and all tumour? Selected healthy and all tumour / selected most varied?
        \item How many of the healthy genes are mutated?
    \end{todolist}
    \item [\wontfix] Gene Ontology plots (?)
    \item  [\wontfix] What are the shared pathway? Are these scattered as in the P0 experiments?
\end{todolist}



