
\chapter{Advancing the network approach} \label{s:N_II}


\vspace{3mm}
% \noindent\rule{17cm}{0.2pt}
\fbox {
    \parbox{\linewidth}{
      \begin{itemize}
        \item Non-tumour network representation informs MIBC subtyping
        \item Mutation burden 'gradual' integration
        \item Hierarchical SBM for community detection
        \item A new integrative Module Evaluation Value (iMEV)
      \end{itemize}
    }
}
\vspace{3mm}


\section{Overview}

% Presented the previous work on the network
The previous two chapters introduced the first iteration of the integrative network pipeline developed to stratify the \acrfull{mibc}. Data integration occurred at three different stages: 1) non-tumour data at the network level, 2) mutation burden at the edge weights, and 3) prioritisation of \acrfull{tf} at the degree level. All three integrative methods were successfully used and validated through different graph metrics, but from a biological stance, the most successful strategy resulted from modifying the number of edges, covered in the selective edge pruning \cref{s:N_I:sel_pruning}.


% Introduce the new network pipeline as an advanced version
Building on the previous work presented, this chapter refines the network pipeline through several key improvements: 1) integrating all three datasets—P0, Abs-Ca, and UD—to construct a comprehensive network representation, 2) introducing a new weight modifier that gradually incorporates mutation burden, 3) proposing an integrative module evaluation value (iMEV) that considers gene expression across both tumour and non-tumour datasets, and 4) employing the \acrfull{hsbm}, which is not constrained by a predefined number of communities. This four pipeline upgrades allow for a more flexible and detailed analysis of both the tumour and non-tumour samples.


% Introduction to the limitations - Non-tumour dataset and MEV
\Cref{s:p0} showed that using only the gene expression from the freshly isolated cells (P0 dataset) did not offer a sufficient representation of the non-tumour data, while the weight modifiers in \cref{s:N_I:tum} were not powerful enough to produce a change in disease stratification. When using the non-tumour data to build the network, the pipeline selects the most connected genes in the communities, which are then used with the \textbf{tumour} expression to the Module Evaluation Value (MEV). This means there is a decoupling between the data used to build the network and the one used for tumour stratification. The limitation is addressed in the new version of the MEV introduced, which considers both datasets, introduced in \cref{s:N_II:iMEV}.


% Weight reward changes
Previously in the thesis, two strategies were explored, either rewarding or penalising the mutated genes. The two modifier functions were designed to double the weight value (reward) or to reduce it close to 0 (penalised) for the most mutated genes and leave it unchanged for the rest, as seen in \cref{fig:N_I:modifiers}. \Cref{s:N_I:tum_stratification} showed the effects on the network metrics, but the modifiers had little effect on the tumour stratification\footnote{See the Sankey plot Reward vs Penalised in \cref{fig:N_II:modifiers_comp}}. A possible explanation for this is the lack of a gradual increase in the weights for the genes that have a 'medium' mutation burden and that the number of mutations is not normally distributed, leading to their effects not being picked up by Module Connection (ModCon), see \cref{fig:N_I:tum_leiden_modifiers}. This chapter addresses this limitation by introducing a new weight modifier, which is a Sigmoid function, offering a gradual weight increase. This chapter discuss only on the reward function as the penalised modifier had poorer performance in previous work, this also allowed a more comprehensive analysis of the data integration.


% Community detection - hSBM and SBM
The \acrfull{sbm} and Leiden are the main community detection methods applied in the network pipeline. The SBM is preferred over Leiden as it is less prone to identify communities from noise. The SBM has a theoretical limit on how many groups it can find in a network, as explained in \cref{s:lit:sbm}. To address this, the hierarchical version of SBM is used in the new network pipeline. This comes with higher running times and computational costs but is still manageable on a medium-performance laptop.

There are four major changes brought to the network pipeline: a larger non-tumour dataset used to build the graph, a new sigmoid reward weight modifier for mutation burden integration, a new MEV score, and the application of \acrfull{hsbm} for community detection. The methods section (\ref{s:N_II:methods}) covers in detail the changes brought to the pipeline, and the validation section (\ref{s:N_II:validation}) outlines the work to check the improvements made. The first results section (\cref{s:N_II:std_net}) focuses on the network obtained from the non-tumour dataset without mutation integration and how it can be used to find new subtypes in both cancerous and non-tumour samples. The second results chapter centres on the reward network (\cref{s:N_II:rwd}), incorporating the mutations and focusing on the stratification of the MIBC.


\section{Aims}

This chapter aims to advance the previously introduced network method by:
\begin{itemize}
    \item Sigmoid Weight modifier - Edges weight modifier improvement - a sigmoid function is used to shape the changes of the correlation values
    \item iMEV - Both tumour and non-tumour gene expression contribute to the values that are clustered 
    \item All non-tumour datasets - Three different gene expression datasets are used to build the network: differentiated (ABS-Ca), undifferentiated and freshly isolated samples (P0)
    \item Hierarchical Stochastic Block Model (hSBM)
    \item Develop our own Software package which facilitates the data integration into a network
\end{itemize}

\subsection*{Datasets}

% Not sure if I need to re-mention this
As in the previous sections, the mutation and gene expression data from the \acrshort{tcga} dataset is employed as the tumour dataset. For the non-tumour dataset, the mRNAseq data from the \acrfull{jbu} generates most of the networks in this chapter. This consists of samples that are \acrlong{absCa} differentiated, undifferentiated (UD) and freshly isolated cells (P0). The list of TFs is taken from the Human Transcription Factors work by \citet{Lambert2018-el}.

Using all the bladder samples available (UD, ABS-Ca, and P0) ensures gene expression from both differentiated, undifferentiated, and freshly isolated cells. This ensures that the gene abundance represents both the Basal and Luminal phenotypes seen in MIBC. This is an important change compared to the P0 network from \cref{s:p0}, which contains only differentiated samples.


\import{Sections/Network_II/}{methods.tex}

\import{Sections/Network_II/}{validation.tex}

\import{Sections/Network_II/}{standard_net.tex}

\import{Sections/Network_II/}{reward_net.tex}

\section{Discussion}

% Introduction and summary
The work present in this chapter results from applying the improved version of the network pipeline to stratify the MIBC. It includes building a network from the non-tumour RNAseq data, integrating TFs via selective edge pruning and mutations via the reward edge weight modifier, and both the expression from tumour and non-tumour via iMEV. Throughout the chapter, the standard and reward networks were tested and validated computationally and through biology's lens. 

% Standard and reward network findings
The analysis of the standard network showed that the P0 \& ABS-Ca datasets can be split into two different subgroups, providing further insight into non-tumour biology and more malignant datasets. The communities responsible for the splits can be traced back to the network, showcasing the transparent nature of the graph representation. The reward network highlighted a subset of genes with a high mutation burden and are strongly and highly co-expressed in the non-tumour dataset, confirming the data integration was successful. The MIBC stratification found a Basal group with a poor survival prognosis and a Luminal with relatively good survival.

%%% General Remarks
% Many ENSG genes - due to the non-tumour 
Many Ensembl genes were found throughout all the subgroups, either of the MIBC or the non-tumour dataset. The presence of many unnamed genes can be explained by using the non-tumour dataset to form the molecular representation. It is also noted that gene filtering keeps the ones expressed in at least 90\% of the samples. This means that there are many 'core genes' that play an important role in bladder differentiation and the MIBC subtypes, indicating the potential for novel bladder molecular biology. 

% Both find Basal and Luminal signals
In the first results section, the MIBC is stratified using the standard network without any integration of the mutation but using the iMEV, which uses the expression from both the tumour and non-tumour dataset. The subtyping with the reward network resulted from integrating the mutation burden through edges' weights. In both instances\footnote{And as a matter of fact in all the previous work done in this PhD and the other classification}, the two 'signals' from Basal and Luminal are present throughout the subgroups; see \cref{s:lit:datasets_used}. The same signals are present in this project's previous sub-groupings and in previous MIBC classifications \citep{Robertson2017-mg,Kamoun2020-tj,Marzouka2018-ge}.

% No IFNG 
The previous Basal split based on the \acrlong{ifn} is not found when using the non-tumour graph representation. This is explained by the fact that the samples in the non-tumour dataset do not show an \acrshort{ifn} response a response specific to bladder cancer patients that Bacillus Calmette–Guérin (BCG) treatment \citep{Baker2022-bj}. However,  in the clustering analysis of the iMEV scores it was found that one of the immune infiltrated subgroups exhibit basal phenotype.

% Main finding: network approach works
The main finding of this chapter is that the (standard) non-tumour network leads to new divisions in both the ABS-Ca and P0 datasets, as well as in TCGA's MIBC cohort. These divisions can be traced to a few communities that consist of Ensembl genes. Additionally, when the mutation burden is integrated, resulting in the reward network, 122 highly connected genes are identified, which have both a high mutation burden and strong correlations with many other genes. This demonstrates that the network is capable of isolating biologically relevant genes.

The following chapter discusses the implications of the network approach developed, the results and their value to the field, as well as the limitations and future work to further improve the methods.



\newpage

% \section{Novelty}

% This chapter brings the following contributions to the field:
% \begin{itemize}
%     \item A network representation of the non-tumour dataset, applied to stratify \acrfull{mibc}.
%     \item A simple method to integrate mutation burden into the graph via the reward modifier.
%     \item Identification of a subset of genes and communities with a high mutation burden, which are highly and strongly co-expressed with other genes.
%     \item Utilisation of \acrfull{hsbm} in the network pipeline, which is able to find small communities in a network of 5000 nodes.
%     \item Identification of seven new MIBC groups, within which Basal and Luminal 'signals' are observed.
%     \item Discovery of new Basal and Luminal subdivisions: the former with a low survival rate and the latter with the highest survival rate over 5 years. Both are characterised by unnamed genes.
%     \item Identifying new splits in the Abs-Ca and P0 datasets
%     \item An explainable computational solution which allows tracing the end result (i.e. subtyping) to network communities and implicitly to a subset of genes
%     \item Across the analysis, many differentially expressed genes were found to be Ensembl genes, highlighting the potential for discovering new biological insights.
%     \item Development of a Python package that creates the network, which is planned to be released to the research community.
% \end{itemize}



% \section{Limitations and future work }

% % largeer dataset
% Throughout the work presented in this chapter it is clear that the community detection is capable of finding small communities in the reward network. This challenges the common methods for filtering the expressed genes, where  the aggressive gene filtering at the beginning of the pipeline, only 5000 genes are used to build the graph. In the next iteration of the network pipeline a less restrictive input dataset, for example of 7500 - 10000 genes, can be used. The advantage of not imposing any filtering is that it decreases the chances to miss important genes. For example, work in the JBU lab found that \textit{RARG} has an important role in tissue differentiation which is ranked as in mid 5000 by the gene selection used throughout the project. The risk of using more genes is that it will be too much noise to uncover the signal, but the the small communities found in the reward network give confidence that this can be overcome.

% % iMev with mutation representation
% %  include the point mutations
% Chapter \cref{s:N_II} introduced an integrative \acrfull{mev} which considers the gene expression from both the network representation (i.e. non-tumour) and the dataset which is stratified (i.e. TCGA's MIBC cohort). However, this does not consider the mutation in each sample nor the type of mutations. In future work, this kind of information needs to be integrated in order to achieve a full integration of the mutation. It is probably a similar approach to the gene expression, but instead of using continuous values, maybe discrete values will be more indicative.

% % iMEV and ModCon
% Another important limitation of the iMEV and ModCon is that the community imbalance is not accounted for. The same number of genes (100) are selected as representative of the block regardless if the community is large or not. The decision to include 100 genes for each was made in order to have sufficient high to compute MEVs and to account for potential differences in the genes mismatch between tumour and non-tumour dataset. This values was set more from experimental observation without a rigorous testing and study of the effect on the end MIBC stratification. Thus, in future work this gap needs to be addressed and the size imbalanced to be accounted for.

% % Gene selection
% Through the thesis it has been alluded that the gene filtering used in this project is more aggressive by removing all the unexpressed genes in >10\% of the samples. This is a different approach then the one taken from the literature its potential implications are presented in the discussion section (\ref{}). For future work, a direct comparison between the two gene filtering methods and their implications in building a network and to the MIBC stratification.

% % Direct comparison with Leiden and hSBM
% Community detection play an important role into the network pipelines developed in this project but also to other bioinformatics work (e.g. single-cell analysis). The research present in this project shows that the Stochastic Block Models are more powerful than Leiden. The SBMs are capable of finding hierarchical structures in the communities, wobble genes as well as finding small communities. The only disadvantage is their large computational time which is still manageable. The comparison between the two classes of community detection will be completed if Leiden would be applied to with the sigmoid modifier and check the communities found with it against the ones discovered with hSBM. This last comparison with the other work done in this project will complete the comparison between Leiden and SBM in the co-expressed networks.

% % Female imbalance
% The non-tumour dataset contains 88 samples which is predominantly females with a few male-paediatric samples. This may represent a limitation of the work done in this project as the MIBC is a male dominated disease. Nevertheless, the P0, UD and Abs-Ca dataset used is one the largest non-tumour dataset and there are plans to open to the research community in the near future.


% \textbf{This might be explained by the MIBC samples have already a 'malignant signal' in the gene expression which is only then amplified by integrating the mutation burden. In contrast with this, the P0 samples still resemble the bladder tissue without the 'malignant signal' and there is no signal to amplify. The freshly isolated samples do not exhibit undifferentiated phenotype which is present in the basal tumour another reason for the small changes across the P0 networks.}

