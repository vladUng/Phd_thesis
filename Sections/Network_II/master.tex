
\section{Network II}

Changes in the network approach:
\begin{itemize}
    \item Community detection - hSBM
    \item Chosen a TF - 6
    \item Reward modifier
    \item MEV take in account both the expression in non-cancerous and tumour
    \item MEV with the mutations - what are the experiments here?
    \begin{itemize}
        \item There is a section in the healthySBM notebook where I am doing a quick comparison (iMev + mutation)
    \end{itemize}
    \item Stating the goal of the method
\end{itemize}

\hfill

Experiments performed:
\begin{itemize}
    \item Exploration of the communities in the standard
    \begin{itemize}
        \item Abs-Ca, UD and P0
        \item Abs-Ca split
        \item Stratification of the benign uropathies 
        \item Cluster-tree and their significance
    \end{itemize}
    \item Community and biological functions 
    \begin{itemize}
        \item Apply GO and GSEA to explore their biological functions
        \item Detail the methods tried and the results found
        \item Maybe link it back to PGCNA and SCES scores and say that we weren't successfull
        \item Conclude that it didn't work out
    \end{itemize}
    \item Reward network analysis
    \begin{itemize}
        \item AHR and finding the correlation with SB's study
        \item other small communities with high connectivity
        \item comparison with small vs large communities
        \item PPARG and RARG enrichment
        \item Interferon search
        \item Looking at the genes which are highly connected
        \item Wobble genes - where do they lie on?
    \end{itemize}
    \item Gene selection - v3 and V4
    \begin{itemize}
        \item Differences in the gene selection
        \item comparing with the varied genes
    \end{itemize}
\end{itemize}

Essentially there are three biological studies that I can relate: 1) Interferon study 2) AHR study and 3) PPARG and RARG study.

\import{Sections/Network_II/}{h_derived.tex}

\newpage
