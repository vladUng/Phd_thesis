\chapter{Discussion}


% Aggressive gene filtering
The full-extent of the implications of using aggressive gene filtering was not fully explored or presented in this thesis as is still a research in progress venue. However, from the work done in this project it is evident that both permissive and aggressive filtering strategy have the potential to reveal new biology. The former by highlighting the most varied genes across the dataset while the later by focusing on the small disruptions of the 'core' or the immune-related (e.g. \acrlong{ifn}) genes. Studying and understanding the disruption of the 'core' genes was crucial in the case of building the non-tumour network representation in \cref{s:N_I, s:N_II}. This means, that in the case of using a 'healthy' molecular representation it is important to apply the aggressive filtering.
