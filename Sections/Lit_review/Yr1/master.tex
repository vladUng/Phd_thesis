

% \import{Sections/Lit_review/Yr1/}{other_methpds.tex}

\import{Sections/Lit_review/Yr1/}{computational.tex}

\import{Sections/Lit_review/Yr1/}{networks.tex}

\import{Sections/Lit_review/Yr1/}{biological_analysis.tex}

% \import{Sections/Lit_review/Yr1/}{biomedical.tex}


\subsection{Summary} \label{s:lit:choosing_ml}

\Cref{s:lit:ea_overview} covered the basics of the evolutionary approach, and in both \ref{s:lit:ann_overview} and \ref{s:lit:autoencod_overview} sections, the connectionists' view of ML was presented. These two methodologies address different types of problems: EAs are more suited for search problems, while ANNs excel in classification tasks. ANNs operate by optimising the configuration of weights and network topology, which can be seen as a search/optimisation challenge to find the right network configuration. 

Consequently, there has been significant research into combining these two approaches in what is termed Neuroevolution. This approach has been employed to process mutation data, as discussed in \cref{s:lit:mutations}. In these computational methods, EAs are used to establish the optimal initial configuration of the ANN, which is then used for training. Starting from a state other than random helps the ANN to circumvent the local minimum problem and enhance classification performance. Additionally, EAs are employed to determine the best network topology, including the number of layers, units, and their connections.

From an engineering perspective, the problem studied in this project is characterised by a large number of features and a small number of samples. It is an unsupervised learning type problem, which needs to determined the groups in an unlabelled data. There are multiple levels of information available for each gene which includes gene expression, mutations and epigenetic data. Therefore, it is important to choose a computational method that satisfies the project's aims and objectives. The next section aims to review the work done across the genomics field by examining various approaches depending on the datasets used.
