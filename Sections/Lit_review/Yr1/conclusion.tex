% \section{Practical} \label{s:practical}

% \subsection{Setting up}

% % the fact that experiment can be run easily 


% \subsection{TCGA replicating}
% Work of the pre-processing I've done so far? but not sure how relevant this is or if it deserves a dedicated subsection


\subsection{Summary}

The purpose of chapter \ref{s:computational} was to support the materials covered in the previous chapter which presented an overview of the effort done in the field of genomics. It started with the progress in analysing the RNAseq data to which the hierarchical clustering is the Consensus (\citet{Kamoun2020-tj}) approach for subtyping the bladder cancer. The section was followed by a survey of the methods used in processing the mutation data with models such as MDPFinder \citet{Zhao2012-wj} or Dendrix \citet{Vandin2012-cf}. The prevalent approach in these studies is to apply different methods either stochastic or greedy to find the subset of relevant mutations. Next, it's given an overview of graph/network theory approaches to genomics, where the work is aiming to find and weigh the connections between genes. Followed, by the section of the Suggested pipeline which is a catalogue of proposed flows in analysing the genomic data, where the researchers proposed a series of steps to apply different methods. Lastly, we have looked at the Deep Learning methods to genetics data, from which we've found out that Autoencoders are a veritable candidate for reducing the dimensions of the data. A summary of the models reviewed can be seen in Appendix \ref{ap:tables_models}, where Table \ref{tab:data_used} along the approaches used, In Table \ref{tab:approaches} the same models are seen but with the datasets and goals of the research.

A first impression might be that there is a lot of effort in processing multi-omics data, and it would be the right impression, for good reasons; cancer is still one of the leading causes of death across the globe. However, the vast majority of the work done in the field is to find the driver mutations or genes that cause a type of cancer and less effort in defining the subtypes of cancer. One possible reason for this is that defining subtypes require biological knowledge and that needs to be specific to the type of cancer studies. The Consensus \citet{Kamoun2020-tj} have defined 6 subgroups for bladder cancer by using only Gene Expression data and validated with the domain knowledge. However, wouldn't adding mutations data increase the confidence in understanding the different types of cancers as well as leading to better clinical trials? Criticism to the Consensus classification is that the mRNA subtypes have not drastically helped clinical translation and that many existing drugs are based on driver gene mutations and protein markers from histology. Therefore, there is a need for the molecular subtypes to explain the biology, probably the ill behaviour of the tumours, and find out what drugs are most useful. 

% This follows naturally from the work reviewed in Section \ref{s:mutations} which focused on processing the mutation data. 

Despite predicting survival rate or tumours events, \citet{Chaudhary2018-qj} and \citet{Ma2019-hk} suggested that integrating mutations and gene expression will lead to better predictions. This serves as an encouragement to one of the aims of the project which is to improve bladder subtyping by including mutation information. Through the MDPFinder \citet{Zhao2012-wj} we have some indication that \acrfull{ea} might be a useful tool in processing genomics data. Their main advantage over the other approaches is that they enable a wider-search space, and this may lead to finding genes that were previously overlooked\footnote{Remember the example of TP53 gene expression}. The Autoencoders are a good candidate for replacing the more 'traditional' dimension reduction techniques as it conserves better the information and seems to accommodate well different data types. 

To sum it up, the project aims to improve the \acrfull{mibc} subtype classifications by incorporating the mutation data into a new model. In the literature survey, we found out there is a Consensus for subtyping based on mRNA which was using hierarchical clustering and there is an effort in finding the driver genes. However, we were not able to find models that use mutations to enforce the subtyping based on gene expression, which we hypothesis will be more biologically relevant and lead to targeted treatments. Autoencoders seems to be a useful tool to be used in the pre-processing stage, while EA may be suitable for analysing the gene network and may represent a way to integrate mutations in the analysis. The following subsection is describing the Aims \& Goals of the project and to the submission point, there is a Gantt Chart with the project's timeline.

% The next sentence doesn't try to answer this question, so it's left hanging. This is a perfect opportunity to briefly mention how the mRNA subtypes have not drastically helped clinical translation, and that many existing drugs are based on driver gene mutations and protein markers from histology. We, therefore, need the molecular subtypes to explain the biology, to say what's gone wrong, and therefore what drug(s) might be most applicable.


% It is worth mentioning that most of the research presented in Section \ref{s:multi-view} is not directly relevant to this PhD project as most of the datasets are on the protein/gene interaction, and this project focuses on the mutations \& Gene Expression. On top of that, the workflows presented in Section \ref{s:pipelines} tend to have a large number of models involved at different steps in the process, but the papers presented there tend to lack of clinical applications. Nevertheless, the work presented in both functions ensures a better representation effort in the genomics landscape. Lastly, the work presented throughout this report is not exhaustive, and during the PhD project, I will constantly fill the gaps, explore new territories, and define better this landscape.


