\subsection{Spiking Neural Networks}

\begin{itemize}
\item SNN and their applications over conventional ML
    \begin{itemize}
      \item time-dependent 
      \item encode more information
      \item Biomedical applications (brain analysis)
    \end{itemize}
\item History of Spiking Neural Networks -> (is this needed?)
\item The most popular Neuron Models. I will present briefly their features and how they differentiate from each other. 
    \begin{itemize}
      \item Hodgkin Huxley (HH) 
      \item Leaky-Integrate Fire (LIF) 
      \item Izhikevich 
    \end{itemize}
\item Applications of SNN in biomedical. 
\begin{itemize}
  \item Understanding the brain
  \item real-life applications?
  \item some applications to microarray data
\end{itemize}
\item Below are some ideas that are important in the context of SNN but I wouldn’t dedicate their own sections at the moment as we don’t know what we need. 
    \begin{itemize}
     \item Network topology 
      \item Learning rules 
      \item Encoding/Decoding
      \item Neuromorphic, this is not essential for the literature review but should be included in the main thesis 
      \item Heterogeneous firing. I found this intriguing as in my view it represents an additional dimension to encode information in a network. The gist of this is that the pre-synaptic firing can arrive at different times to the postsynaptic neuron. Besides, there is some work introducing dendritic firing which is linked to heterogeneous firing as each dendrite has its own travelling time for the electric impulse. My understanding is that even though the new models introduce another dimension for encoding information the computational cost is not significant. 
     \end{itemize}
 
\end{itemize}
