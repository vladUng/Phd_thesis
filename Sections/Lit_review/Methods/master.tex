\section{Methods and Data} \label{s:lit:computational}

\vspace{3mm}
% \noindent\rule{17cm}{0.2pt}
\fbox {
    \parbox{\linewidth}{
      \begin{itemize}
        \item Datasets used: \acrshort{tcga} and healthy from \acrlong{jbu}
        \item Clustering methods and the metrics used
        \item Networks and describing a network
        \item Tools for statistical analysis of gene expression
      \end{itemize}
    }
}
\vspace{3mm}

% Briefly mentions the other types of learning and why are not suitable
\acrfull{ml} methods are generally categorised into three types: supervised, semi-supervised, and unsupervised learning. In supervised learning, data are annotated with known labels, allowing models to learn associations between input features and predefined output classes. This is often considered the most straightforward case, as the outcome is known and models can be evaluated against clear targets; however, it depends on the availability of accurately labelled datasets, which are often costly and time-consuming to produce. In semi-supervised learning, models are trained using a small amount of labelled data alongside a larger pool of unlabelled data, guided by predefined rules or assumptions. This approach is useful when full labelling is impractical, but some structural information about the data is still available.

In contrast, omics datasets typically consist of a relatively small number of samples and a very large number of features, making them less suitable for supervised or semi-supervised learning. In particular, for disease stratification tasks—such as identifying novel subtypes in mRNA-sequencing data—unsupervised learning is more appropriate, as it does not require predefined labels or assumptions about the number or nature of subgroups. This project therefore employs unsupervised methods to analyse, learn from, and ultimately discover new subtypes of \acrlong{mibc}. In this context, clustering is the starting analytical approach, enabling the identification of biologically meaningful groups, which are then interpreted using clinical knowledge and external validation techniques.

Among unsupervised learning techniques, dimensionality reduction and clustering are particularly prominent in genomics, especially for disease subtyping based on mRNA sequencing. For example, in the stratification of the MIBC cohort from TCGA, \citet{Robertson2017-mg} pre-processed gene expression data, applied non-negative matrix factorisation (\acrshort{nmf}) for dimensionality reduction, and subsequently performed hierarchical clustering; further details are provided in \cref{s:lit:tcga_mibc}. A similar approach is used in this project (\cref{s:clustering_analysis}), with slight modifications in the pre-processing pipeline: principal component analysis (\acrshort{pca}) is used for dimensionality reduction, followed by K-means clustering for subtype identification.


While clustering and dimensionality reduction serve as important analytical tools, the central methodological framework of this thesis is grounded in network theory. Network-based approaches model gene relationships in a manner that more closely reflects the underlying biological processes of the bladder, in both tumour and non-tumour contexts, than traditional statistical methods. By constructing co-expression networks from RNA-seq data, this project identifies patterns of interaction between genes that can reflect functional organisation and regulatory structure. Community detection algorithms are applied to these networks to uncover groups of co-expressed genes, offering insights into both tumour and non-tumour transcriptomes. Moreover, network analysis enables the integration of multiple data types and the prioritisation of biologically relevant genes—particularly transcription factors—through connectivity-based metrics. This systems-level perspective provides a richer context for interpreting molecular subtypes and complements the clustering analysis performed earlier in the workflow.


Therefore, this part of the thesis serves as a foundation for all the results chapters, where various techniques were employed to assess clustering configurations, evaluate the biological significance of newly discovered \acrshort{mibc} subtypes, and define their molecular characteristics. The chapter starts by covering the datasets used in the project (\cref{s:lit:datasets_used}), followed by introducing the clustering models (\cref{s:lit:clustering}), the metrics used to assess their performance (\cref{s:lit:clustering}), and dimension reduction techniques (\cref{s:lit:dim_red}). It also explains the concepts of networks/graphs in \cref{s:lit:graph_overview} and the methods for describing them in \cref{s:lit:net_metrics}. The \cref{s:lit:gene_analysis} covers the main tools for analysing MIBC subgroups including \acrfull{dea}, which contains the Volcano and Pi plots (\cref{s:lit:dea,s:lit:pi}), \acrfull{gsea} (\cref{s:lit:gsea}) and Kaplan-Meier in \cref{s:lit:survival}.


\import{Sections/Lit_review/Methods/}{datasets.tex}

\import{Sections/Lit_review/Methods/}{computational.tex}

\import{Sections/Lit_review/Methods/}{networks.tex}

\import{Sections/Lit_review/Methods/}{biological_analysis.tex}

% \import{Sections/Lit_review/Yr1/}{biomedical.tex}
% \import{Sections/Lit_review/Yr1/}{other_methpds.tex}


\subsection{Summary} \label{s:lit:choosing_ml}


This section covered the tools used to analyse the gene expression datasets in this project. It is important to note that this section was not a literature review of these techniques but rather a background for the methods used throughout the project. \Cref{s:lit:multi-omics} covers different approaches to analysing omics data, \cref{s:lit:nets_bio} presents research on using networks within biology, and \cref{s:lit:comm_detect} focuses on detecting groups within networks.

From an engineering perspective, the problem studied in this project is characterised by a large number of features and a small number of samples. It is an unsupervised learning type problem, which needs to determined the groups in an unlabelled data. There are multiple levels of information available for each gene which includes gene expression, mutations and epigenetic data. Therefore, it is important to choose a computational method that satisfies the project's aims and objectives to integrate multiple data types. The next section aims to review the work done across the genomics field by examining various approaches depending on the datasets used.
