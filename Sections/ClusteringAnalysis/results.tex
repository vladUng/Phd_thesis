\section{Results: the groups molecular biology} \label{s:cs:bio_interp}


% Introduction our decision that we are going to use the work from Simon
The first part of the chapter found the clustering configuration, specifically K-means with five cluster, using a purely computational approach supported by various clustering metrics. In this second part, the focus shifts to understanding and defining the biological aspects of the derived MIBC subtypes as well as biologically validating the new groups found. This involves a comparative analysis with other classifications in the literature which includes TCGA, Lund and consensus classifiers \citep{Robertson2017-mg, Marzouka2018-ge,Kamoun2020-tj}.

To elucidate the specific biological pathways associated with each MIBC subgroup, established methods such as \acrfull{dea} and \acrfull{gsea} are employed. These analyses are instrumental in identifying distinct gene markers and pathways that characterise the subgroups, providing a deeper insight into their biological underpinnings.

% TCGA and consensus classification
By comparing with TCGA and consensus classifications \citep{Robertson2017-mg,Kamoun2020-tj} it can be noticed in \cref{fig:cs:ifn_all} that the Ba/Sq is split into two a smaller and larger subgroups, there is a \acrfull{ne}, a \acrfull{lump} group and a combine Luminal Infiltrated/Non-specified (Lum Inf/NS). The names are inferred from the previous classifications for practical reasons to ease the comparison.


\begin{figure}[!t]
    \centering
    \begin{subfigure}[!t]{1.0\textwidth}
        \includegraphics[width=1.0\textwidth,keepaspectratio]{Sections/ClusteringAnalysis/Resources/discussion/inf_comp.png}
        \caption{All the groups derived with K-means (K=5)}
        \label{fig:cs:ifn_all}
    \end{subfigure}
    \begin{subfigure}[!t]{1.0\textwidth}
        \includegraphics[width=\textwidth,keepaspectratio]{Sections/ClusteringAnalysis/Resources/discussion/inf_3_basal.png}
        \caption{Further split of the Basal group into 3 subtypes, using the work from \citet{Baker2022-bj}}
        \label{fig:cs:ifn_three_basal}
    \end{subfigure} 
    \caption[MIBC groups comparisons with previous classifications]{Sankey plot comparing TCGA classification \citep{Robertson2017-mg}, the \acrlong{ifn} split from \citet{Baker2022-bj}, groups from cluster analysis (\cref{s:clustering_analysis}) and the Lund classifier \citep{Marzouka2018-ge}. This plot highlights that the two basal splits found through our clustering are a group of samples that have high \acrshort{ifn} (small group) and a mixed response (Large Ba/Sa). }
    \label{fig:cs:ifn_comp}
\end{figure}

% IFNG
The groups derived in this section are also compared with the \textit{in vitro} work \citet{Baker2022-bj} from the Jack Birch Unit (JBU) focused on splitting the Basal/Squamous group by the tumour's \acrfull{ifn} response: no response and a high response. The authors identified that basal classified samples with a high \acrshort{ifn} have a better survival prognosis compared to the ones with no immune response; the \acrshort{ifn} signature response is derived from the gene expression of a subset of genes found by the authors. In \cref{fig:cs:ifn_three_basal} it can be clearly seen that the Small Ba/Sq group contains a large portion of the the \acrshort{ifn} response group from \citet{Baker2022-bj}, while the large basal group a mixed of low and high \acrshort{ifn} samples. Taking the 3 main 'chords' of samples going from the \acrshort{ifn} to the Basal groups (KMeans), the Basal subtype can be further split into 3 as seen in \cref{fig:cs:ifn_three_basal}. The association of the Basal subgroups it is also supported by the Lund classifier \citep{Marzouka2018-ge}, where the Small Ba/sq (or High IFNG) contains the samples of the Ba/Sq-Inf and Mes-like from Lund, while the other two basal (in Basal type) are either Ba/Sq or UroB by Lund.

% Introduce the three splits
\subsection{Immune and Stromal infiltration}

% Explaining the ESTIMATE score
To further validate the immune characteristics of the subgroups, the ESTIMATE score was employed \citep{Yoshihara2013-wq}. The authors assesses tumour purity based on immune and stromal infiltration by calculating the gene expression of a subset of genes that are known to be involved in the immune response and associated with stromal cells. The ESTIMATE, along with stromal and immune scores, were validated using DNA Copy Number Analysis and computed for TCGA cohort by the authors \citep{Yoshihara2013-wq}.

% Check the three splits along the immune infiltration, ESTIMATE, and stroma score
By using both K-means (K=5) and the \textit{in vitro} work \citep{Baker2022-bj}, the three basal groups can be classified into three different groups proportional to their \acrfull{ifn} response: Low, Medium, and High \acrshort{ifn} Basal tumours. An initial confirmation of the basal immune heterogeneity is show in \cref{fig:cs:tumour_purity}. Across all the plots, the Y-axis is represented by the \acrshort{ifn} score from \citet{Baker2022-bj}, and the X-axis represents one of the three scores (immune, stromal and ESTIMATE) from \citet{Yoshihara2013-wq}, while the marker size is proportional to the infiltration score from TCGA's metadata. The first scatter plot shows that High \acrshort{ifn} has the largest immune infiltration and \acrshort{ifn} response, while Low \acrshort{ifn} is at the opposite end of the spectrum, with Medium \acrshort{ifn} in between.

\Cref{fig:cs:stroma_basal} shows the relationship between the three Basal subgroups and stromal infiltration. The spectrum is not as clear as in the immune plot, as both Low and Medium \acrshort{ifn} have a low stromal score, but the differences with the other Basal groups are clear. Lastly, the two scores are combined in the ESTIMATE scatter plot, shown in \cref{fig:cs:estimate_basal}, displaying the tumour purity of the samples in the three groups. The figure illustrates the spectrum-like nature of the Basal subtypes and their relationship, with Low \acrshort{ifn} showing the highest purity, followed by Medium and High \acrshort{ifn}, the latter having the highest stromal and immune cell impurity.


\begin{figure}[H]
    \captionsetup{font=small} 
    \centering
    \begin{subfigure}[!t]{0.89\textwidth}
        \includegraphics[width=\textwidth,keepaspectratio]{Sections/ClusteringAnalysis/Resources/discussion/Immune_spectrum.png}    
        \caption{Immune Score}
        \label{fig:cs:immune_basal}
    \end{subfigure}
    \centering
    \begin{subfigure}[!t]{0.89\textwidth}
        \includegraphics[width=\textwidth,keepaspectratio]{Sections/ClusteringAnalysis/Resources/discussion/Stroma_spectrum.png}
        \caption{Stroma Score}
        \label{fig:cs:stroma_basal}
    \end{subfigure} 
    \centering
    \begin{subfigure}[!t]{0.89\textwidth}
        \includegraphics[width=\textwidth,keepaspectratio]{Sections/ClusteringAnalysis/Resources/discussion/Estimate_spectrum.png}
        \caption{ESTIMATE}
        \label{fig:cs:estimate_basal}
    \end{subfigure}
    \centering
    \caption[New MIBC subgroups and their tumour purity]{The tumour purity as indicated by the immune, stroma, and ESTIMATE scores \citep{Yoshihara2013-wq} for the three basal splits. In all scatter plots, the Y-axis is represented by the \acrshort{ifn} score \citep{Baker2022-bj}, the X-axis by one of the three scores, and the size of the markers is proportional to the infiltration score from \citet{Robertson2017-mg}. Positive values of the score denote infiltration by the type of cells, while negative values indicate their absence. The three figures shows how both the immune and stromal infiltration as well as the tumour impurity (ESTIMATE) increases with the \acrshort{ifn} response of the samples.}
    \label{fig:cs:tumour_purity}
\end{figure}


% Survival 
\subsection{Survival analysis}

The Lund classification and the work on urothelium's \acrshort{ifn} response \citep{Baker2022-bj} split the Basal into two groups; \citep{Marzouka2018-ge} - \acrfull{ba/sq} and \acrfull{ba/sq-inf} and \citep{Baker2022-bj} - one with \acrshort{ifn} response and the other without. In both cases, the two subtypes also have different survival rates, where the samples with immune infiltration had a better prognosis.

The immune response and survival relation are also present in the Basal subgroups derived in this section, shown by performing the Kaplan-Meier survival analysis, see \cref{fig:cs:basal_survival}. The figure displays the Basal group from \citet{Robertson2017-mg} by the green line, and the other lines represent the subtypes found in this section. From the survival rates, it can be noticed how the three basal groups constitute TCGA's Ba/Sq group and that the poorest prognosis is given by the Low IFNG samples, where the survival rate is less than 30\% after 40 months, showcasing the importance of improving the MIBC classifications. 


\begin{figure}[!htb]    
    \centering
    \includegraphics[width=1.0\textwidth,keepaspectratio]{Sections/ClusteringAnalysis/Resources/discussion/survival_basal.png}
    \centering
    \caption[New Basal subgroups and their survival prognosis]{Kaplan-Meier survival analysis of the 3 Basal groups derived in \cref{s:cs:bio_interp} using K-means with K=5 and the basal subgroups by the \acrshort{ifn} response \citep{Baker2022-bj}, and the Ba/Sq survival from TCGA classification \citep{Robertson2017-mg}. The plot shows that TCGA's Ba/Sq group is split into three by our clustering methods each with different survival rates. The Low IFNG having the poorest survival prognosis, with less than 30\% after 40 moths.}
    \label{fig:cs:basal_survival}
\end{figure}

% Basal
\subsection{Basal samples and immune heterogeneity} \label{s:cs:basal_interp}

With different survival rate across the Basal subtypes it is imperative to explore further the subgroups derived from the cluster analysis, pipeline presented in \cref{fig:cs:clustering_pipeline}. \acrlong{dea} was used to characterise each group using the Pi-plots and the \acrlong{gsea} introduced in the Background \cref{s:lit:pi,s:lit:gsea}. 

A summary of the differentiation, basal and luminal markers can be found in \cref{tab:ap:pi_genes_1,tab:ap:pi_genes_2} to which the below analysis is kept referring.

\paragraph*{Low IFNG}

% Low IFNG
The Low IFNG group has the poorest survival prognosis making it the focus of the Basal comparisons. The Pi-plot in \cref{fig:cs:pi_basal} compares the Low IFNG with the Med IFNG on the Y-axis and with the High IFNG on the X-axis. The initial observation from the scatter plot is the sparse presence of markers in the quadrant IV, specific to Low IFNG, indicating that there are a few genes distinctly characterising the Low IFNG with respect to the other Basal subtypes. This is specifically highlighted by the small scale of the Y-axis, where the Med and Low IFNG are compared, showing \textit{GSTM3} as being distinct for the latter.

The negative side of the X-axis in \cref{fig:cs:pi_basal} highlights the genes unique to Low IFNG over High IFNG, where there are several known squamous markers such as \textit{TP53, DSC3} are present. Interestingly, \textit{FGFR3}, a well-documented differentiation-associated marker and extensively studied gene in MIBC, is also positioned on the negative side of the X-axis among \textit{ZBTB7C} which is associated with luminal differentiation \citep{Ramal2024-ha}.

% Unknown genes
Some of the genes on the positive side of Y-axis in \cref{fig:cs:pi_basal}, 
are labelled as 'other', were not included in any of the 'known markers': \textit{DANCR, DAPL1, GATM, GSTM3, MLF1, SYNGR1, GSTM1, MAGI3}. Nevertheless, some of these genes play a role in bladder or other cancers. In an \textit{in situ} study by \citet{Zhan2018-um}, \textit{DANCR} (or \textit{ANCR}), a \acrfull{lncRNA}, was found to play an oncogenic role in bladder cancer\footnote{There is a review by \citet{Wang2021-gn} on the role of \textit{DANCR} in bladder and other cancers.}. According to a review focused on \textit{MLF1}, it showed that the gene may be involved in various types of cancer, including bladder cancer, and could influence the immune response \citep{Li2023-mk}. \textit{GATM} is suggested to be a tumour suppressor gene that influences a type of metastatic renal cancer, as indicated in a transcriptomic study by \citet{Jee2022-wi}. \textit{GSTM3} and \textit{GSTM1} are known to be genetic risk factors for bladder cancer \citep{Schnakenberg2000-cu}. The roles of \textit{MAGI3, MLF1,} and \textit{DAPL1} are less well-known.

\begin{figure}[H]
    \centering
    \begin{subfigure}[!t]{1.0\textwidth}
        \includegraphics[width=\textwidth,keepaspectratio]{Sections/ClusteringAnalysis/Resources/discussion/basal_comp_pi.png}    
        \caption{Basal comparison}
        \label{fig:cs:basal_comp}
    \end{subfigure}
    \centering
    \begin{subfigure}[!t]{1.0\textwidth}
        \includegraphics[width=\textwidth, keepaspectratio]{Sections/ClusteringAnalysis/Resources/discussion/basal_comp_pi_zoom.png}
        \caption{Basal comparison zoom in}
        \label{fig:cs:basal_comp_zoom}
    \end{subfigure} 
    \centering
    \caption[Pi-plot - comparisons between Basal groups]{The Pi plots with the \acrshort{dea} between Low and High IFNG (X-axis) and Low and Medium IFNG (Y-axis). The points in distinct colours representing the 98 \acrlong{tf} (Selected TF - next section \cref{s:N_I:sel_tfs}) and known markers presented in \cref{tab:ap:pi_genes_1,tab:ap:pi_genes_2}. Quadrant IV highlights the genes specific to Low IFNG, with the "Ref point" denoting the 'most' Low IFNG significantly expressed genes, and the top 10 closest points to it are displayed in orange. Quadrant II contains the genes most common to the other two Basal groups.}
    \label{fig:cs:pi_basal}
\end{figure}

% Specific to the low over the high 
When zoomed in the Pi plot from \cref{fig:cs:basal_comp} becomes the figure in \cref{fig:cs:basal_comp_zoom} and on the negative side of the horizontal axis, in \cref{fig:cs:basal_comp_zoom}, it ca be seen the Basal markers are significantly differently expressed in the Low IFNG over the High IFNG group. Most of the genes are Squamous/Basal markers.

\paragraph*{High and Medium IFNG}

% High IFNG and MEDIUM 
In the diagonally opposite quadrant (II) from \cref{fig:cs:basal_comp} are genes involved in both High and Medium IFNG basal groups, which contain considerably more markers than the Low IFNG's quadrant. On the positive X-axis, most of the genes are part of the Immunoglobulin family (\textit{IG-}), and are associated with macrophages and monocytes, all known to be related to the immune response. This confirms that High IFNG is the strongest Basal subgroup associated with the immune response. The \acrshort{ifn} response markers from \citet{Baker2022-bj} are enriched in both Med and High IFNG subtypes, as shown by the purple markers.


% Introduce the other plot
To further investigate the relationship of the Basal groups with the immune response of the tissue, another Pi plot was utilised in \cref{fig:cs:pi_basal_inf}. The Y-axis displays the Pi values generated from the DEA of Low IFNG versus Luminal Infiltrated, while the X-axis represents the DEA between High and Med IFNG subgroups. In this analysis, Low IFNG is considered the MIBC subtype with the least immune response, whereas LumInf is regarded as a non-basal group with high immune infiltration. This scatter plot thus elucidates the tendencies of the three Basal subtypes.

% High IFNG
The first quadrant in \cref{fig:cs:pi_basal_inf} contains markers that are specific to both the LumInf and the High IFNG groups. The family of immunoglobulin genes is common to both groups, the scale on the Y-axis is larger than on the X-axis giving the impression that the IG family tends to the basal group. On the positive X-axis, there are many markers for other immune-related responses, such as B and T cells, as well as monocytes and macrophages. This further reinforces the immune characteristics previously observed for High IFNG and highlights the stronger immune response of High IFNG compared to Med IFNG.


\begin{figure}[H]    
    \centering
    \includegraphics[width=1.0\textwidth,keepaspectratio]{Sections/ClusteringAnalysis/Resources/discussion/basal_inf_pi.png}
    \caption[Pi plot - basal groups and immune infiltration]{The Pi plot with the \acrshort{dea} between Low and High IFNG (X-axis) and Luminal Infiltrated and Medium IFNG (Y-axis). The points are coloured distinctly to represent the 98 \acrlong{tf} (Selected TF - next section \cref{s:N_I:sel_tfs}) and known markers presented in \cref{tab:ap:pi_genes_1,tab:ap:pi_genes_2}. Quadrant III highlights the genes specific to the Low and Medium IFNG Basal groups, with the top 10 closest points to the 'Ref point' displayed in orange. This quadrant emphasises the basal characteristics of the two groups. In the diagonally opposite quadrant (I) are the genes most common to Luminal Infiltrated and High IFNG, highlighting the immune nature of the Basal group.}
    \label{fig:cs:pi_basal_inf}
\end{figure}


% Medium and LumInf
\paragraph*{Relationship to Luminal Infiltrated}

From \cref{fig:cs:pi_basal_inf} it can be observed that there are fewer common genes between the Med IFNG and the LumInf in the fourth quadrant. Some of the genes on the negative side of the Y-axis, which are specific to Med IFNG over High IFNG, are involved in squamous or basal tissue, such as \textit{ZBTB7C, ZNF750, KRT13, KLF5, SPRR1A}. The genes specific to Ba/Sq are even more prevalent in the third quadrant, between Low and Med IFNG, where many of the shared genes are well-known squamous or basal markers: \textit{DSC3, KRT6A, KRT5, TP53}, the \textit{SERPIN} and \textit{SPR} families.

The two figures \cref{fig:cs:pi_basal,fig:cs:pi_basal_inf} support three distinct immune responses across the basal tumours. They highlight the genes specific to each subtype, where High IFNG exhibits strong immune responses, even compared to the LumInf subgroup. Med IFNG is positioned between a high immune response and no response, as seen in the samples from Low IFNG. Basal and squamous markers are evident in both Low and Med IFNG groups, as clearly demonstrated in \cref{fig:cs:pi_basal_inf}, and are supported by the ESTIMATE scores in \cref{fig:cs:tumour_purity}.

The three-way split is further advancing the understanding of basal tumours and their immune response. With prominent markers on the Low IFNG subgroups, novel insights are revealed about the 'basalness' and squamous aspect of the group. In addition, this may represent an opportunity to develop targeted treatments for patients with poor survival prognosis.

% LumInf
\subsection{Luminal infiltrated} \label{s:cs:lumInf_interp}

The previous subsections were focused on understanding the molecular properties of the Basal subgroups, identifying High IFNG to be 'closer' to the Luminal infiltrated. To define the molecular particularities of the luminal infiltrated samples, the \gls{PI-PLOT} was constructed using the DEA results from the comparison of the LumInf group with Low IFNG on the X-axis and with LumP on the Y-axis. The Low IFNG group was chosen to represent the Basal groups as it lacks immune infiltration, thereby highlighting the LumInf properties, while LumP was selected to examine both the shared and different luminal features.

% Introducing GSEA
\acrfull{gsea} was run on the REACTOME (see \cref{fig:cs:ne_gsea}) database using GSEApy  package \citep{Fang2023-ec}. The input for GSEA was computed by ranking the distances to the "Ref point" in quadrant II; see \cref{s:lit:gsea} for more details on the method.

% Commenting on the immune aspect of the Pi plot
The distinct branch of points in the second quadrant of \cref{fig:cs:lumInf_pi}, specific to LumInf, includes the immunoglobulin family, denoting the immune response characteristic of the group. This is further substantiated by the results from the GSEA in \cref{fig:cs:lumInf_gsea}, where most pathways found in Reactome involve genes linked to the immune response, such as \textit{FCGR Activation}, \textit{CD22 mediated BCR regulation}, or \textit{creation of C4 and C2 activators}.


% Positive on the X-axis showing the luminal characteristic
The positive values on the X-axis show genes that are differentially expressed in LumInf compared to the Low IFNG group and exhibit a luminal-like aspect. These markers include genes known as luminal markers (see \cref{tab:lit:tcga_genes}), such as \textit{UPK1A, UPK3A, UPK2, PPARG, GATA3}, confirming the luminal characteristics of the subtype. The negative values on the X-axis are Basal-specific markers, such as \textit{KRT6A} and \textit{KRT5}. The lack of scatter points and the small scale of the Y-axis indicate that the Luminal Infiltrated and Luminal Papillary groups do not share many genes.

This analysis confirms both the immune infiltration and the luminal molecular characteristics of the LumInf group, derived using the cluster methods presented in this section.

\begin{figure}[H]
    \centering
    \captionsetup{font=small} 
    \begin{subfigure}[!t]{1.0\textwidth}
        \includegraphics[width=\textwidth,keepaspectratio]{Sections/ClusteringAnalysis/Resources/discussion/other_groups/lumInf_pi.png}
        \caption{Pi}
        \label{fig:cs:lumInf_pi}
    \end{subfigure}
    \centering
    \begin{subfigure}[!t]{0.9\textwidth}
        \includegraphics[width=\textwidth, keepaspectratio]{Sections/ClusteringAnalysis/Resources/discussion/other_groups/lumInf_reactome_10_top.png}
        \caption{Top 10 GSEA results on the Reactome database.}
        \label{fig:cs:lumInf_gsea}
    \end{subfigure} 
    \centering
    \caption[LumInf-like - molecular overview]{Molecular overview of the Luminal Infiltrated (LumInf) group. The Pi plot shows the genes specific to LumInf from the DEAs against Basal Low IFNG (X-axis) and Luminal Papillary (Y-axis). The points are coloured distinctly to represent the 98 \acrlong{tf} (Selected TF - next \cref{s:N_I:sel_tfs}) and known markers presented in \cref{tab:ap:pi_genes_1,tab:ap:pi_genes_2}. Quadrant II highlights the genes specific to LumInf, which are mostly immune-related markers; the top 10 closest points to the 'Ref point' are displayed in orange. The GSEA plot displays the top 10 Reactome terms which primarily consist of immune pathways (FCCR Activation, CD22-mediated).}
    \label{fig:cs:lumInf}
\end{figure}


% LumP
\subsection{Luminal Papillary} \label{s:cs:lumP_interp}

The LumInf group exhibits some of the molecular properties of a luminal subtype, but the immune response of the samples is notably stronger. In contrast, this' noise' does not affect the tissues from the LumP group. To highlight the molecular characteristics of the LumP samples the group is compared  (i.e. \acrshort{dea}) with Ne on the Y-axis and with Low IFNG on the X-axis; check the Pi-plot from \cref{fig:cs:lumP_pi} and the \acrshort{gsea} results in \cref{fig:cs:lumP_gsea}. This configuration ensures that any immune response in LumP is discernible, as a lack of immune activity characterises both compared groups. It also accentuates the markers of the luminal group, given that Low IFNG and Ne possess distinctly different molecular properties.

% LumP specific genes
The points in the first quadrant (\cref{fig:cs:lumP_pi}) represent genes specific to the LumP group. It can be seen that there are several sets of genes that are directly associated with bladder differentiation, a trait common to LumP tumours, such as: \textit{PPARG, GATA3, UPK3A, FOXA1}. Other genes significantly expressed in the Luminal Papillary group include \textit{VSIG2, PSCA, ADIRF, DHRS2, SPINK1}. Analysing GO, KEGG pathways and performing DEA between NMIBC and MIBC (microarray dataset \href{https://www.ncbi.nlm.nih.gov/geo/query/acc.cgi?acc=GSE13507}{GSE13507}), the authors in \citet{He2021-de} identified \textit{VSIG2} as one of the three markers for muscle invasiveness. \textit{ADIRF} has been associated with differentiation by work from Lund's group \citep{Eriksson2015-lt}. \textit{DHRS2} protein is highly expressed (see \href{https://www.proteinatlas.org/ENSG00000100867-DHRS2/tissue/urinary+bladder}{Human Protein Atlas}), but is not well studied in the context of MIBC. In the microarray analysis with immunohistochemistry from \citet{Rink2013-sv}, it was found that loss of \textit{SPINK1} was associated with tumour recurrence and aggressiveness.

To further support the non-squamous aspect of the LumP group, some genes that are down-regulated in \acrfull{scc} taken from \citet{Knowles2015-mu} are present in the first quadrant. In the opposite quadrant, where the genes specific to Ne and Low IFNG are contained, there are markers specific to the Basal (closer to the negative X-axis) such as \textit{KRT6A, KRT6B, KRT14, KRT5}\footnote{Check \cref{tab:ap:pi_genes_1,tab:ap:pi_genes_2}}. Additionally, up-regulated SCC markers are near the X-axis, indicating that Low IFNG has stronger SCC markers compared to both the NE and the LumP group. Neuronal markers such as \textit{TUBB2B, GNG4, APLP1} are on the negative side of the Y-axis, denoting the Ne-like aspect over the other two groups.

% Explore the genes that are common to both the Low IFNG and NE
In the GSEA plot, the top enriched pathways are down-regulated in the LumP, indicating a strong response from the Ne and Low IFNG groups. The analysis in this subsection highlights the markers for the Luminal groups and reinforces the characteristics of the other two subgroups. It also shows that there are a few genes that are significantly expressed and common to both Low IFNG and NE-like groups, the latter being further researched in the next subsection.

\begin{figure}[H]
    \centering
    \small
    \begin{subfigure}[!t]{1.0\textwidth}
        \includegraphics[width=\textwidth,keepaspectratio]{Sections/ClusteringAnalysis/Resources/discussion/other_groups/lump_pi.png}    
        \caption{Pi}
        \label{fig:cs:lumP_pi}
    \end{subfigure}
    \centering
    \begin{subfigure}[!t]{0.83\textwidth}
        \includegraphics[width=\textwidth, keepaspectratio]{Sections/ClusteringAnalysis/Resources/discussion/other_groups/lumP2_reactome_10_top.png}
        \caption{Top 10 GSEA results on the Reactome database.}
        \label{fig:cs:lumP_gsea}
    \end{subfigure} 
    \centering
    \caption[LumP like - molecular overview]{Molecular overview of the \acrfull{lump} group. The Pi plot shows the genes specific to LumP from the DEAs against the Basal Low IFNG (X-axis) and \acrlong{ne} groups. The points are coloured distinctly to represent the 98 \acrlong{tf} (next section \cref{s:N_I:sel_tfs}) and known markers presented in \cref{tab:ap:pi_genes_1,tab:ap:pi_genes_2}. Quadrant I highlights the genes specific to the LumP group, which consist of luminal markers (\textit{UPKs}); the top 10 closest points to the 'Ref point' are displayed in orange. The GSEA plot displays the top 10 Reactome terms, which shows a lack of enrichment in the pathways found in the Low IFNG and Ne groups.}
    \label{fig:cs:lump}
\end{figure}


\subsection{Neuroendocrine} \label{s:cs:ne_interp}


In TCGA's MIBC cohort, there were 20 samples identified as Neuroendocrine like (NE-like) in the work \citet{Robertson2017-mg}, 11 of which are also classified as Sc/NE-like by the Lund classifier, and only 6 of the initial 20 are confirmed as NE-like by the consensus, which includes more than the two datasets mentioned. This suggests that considerable uncertainties still exist regarding the criteria that define the NE group. The cluster analysis conducted in this section grouped \textbf{32} samples into what appears to contain most of the NE-like groups from both TCGA and Lund datasets \cref{fig:cs:ifn_comp}.

To further explore the NE group derived in this section, the Pi-plot in \cref{fig:cs:ne_pi} displays the \acrshort{dea} comparison of the group with Low IFNG on the Y-axis and with Luminal Papillary on the horizontal axis. These two comparison groups were used as 'representatives' for the Luminal and Basal subtypes as they exhibit the lowest immune infiltration, thus facilitating the observation of any Basal or Luminal tendencies in the NE samples, if present. As can be seen in quadrants I and III \cref{fig:cs:ne_pi}, there are few or no points shared between the studied group and the Basal/Luminal subtypes. This highlights the distinct molecular characteristics of the NE group.


% Genes associated with the NE
The second quadrant in \cref{fig:cs:ne_pi} showcases genes specifically associated with the Neuroendocrine like (NE-like) group, including \textit{TUBB2B} and other neuronal markers such as \textit{MSI1, PLEKHG4B, GNG4, PEG10, RND2, APLP1,}\citep{Robertson2017-mg}. These genes are significantly expressed in the NE-like group, as illustrated by the referential point \cref{fig:cs:ne_pi}, which marks the extremities on the X and Y axes, representing the most pronounced NE-like characteristics from the two comparative analyses. Beyond \textit{TUBB2B}, several genes near this marker hold substantial biological significance, including \textit{TUBA1A, FGFR1, SLC16A1, FXYD6, CTHRC1,} and \textit{PRRX1}.

\textit{TUBA1A} is highly expressed in bladder cancer and has been identified as a potential therapeutic target in the work of \citet{Zhang2019-fk} (see \href{https://www.proteinatlas.org/ENSG00000167552-TUBA1A/tissue}{Human Protein Atlas}). \textit{FGFR1}, part of the FGFR family, is associated with cell proliferation, as evidenced in both \textit{in situ} and \textit{in vitro} experiments \citep{Tomlinson2009-td}. \textit{SLC16A1} has been characterised by \citet{Logotheti2020-ya} as a \acrlong{lncRNA} that enhances cancer invasiveness. \textit{FXYD6}, associated with mental diseases, has shown potential in clinical therapies for a subtype of kidney cancer according to \citet{Gao2014-sq}. In vitro studies suggest \textit{PRRX1} as a therapeutic biomarker for bladder cancer \citep{Huang2022-ez}, with its protein expression being moderately high (see \href{https://www.proteinatlas.org/ENSG00000116132-PRRX1/tissue}{Human Protein Atlas}). Additionally, mouse research on \textit{CTHRC1} indicates that the gene has a role in the proliferation of bladder cancer cells, with high levels correlating with poor prognosis. This protein's high expression is also documented on the \href{https://www.proteinatlas.org/ENSG00000164932-CTHRC1/tissue}{Human Protein Atlas}.

% LumP and Basal
The Low IFNG and LumP comparison groups represent the two major MIBC categories, basal and luminal. The genes in the third quadrant include those common to both groups, while those on the axes are specific to each subtype. On the positive vertical axis, genes such as \textit{UPK2, GATA3, ADIRF, VSIG2} are found, which are known to be associated with tissue differentiation or luminal biology. On the horizontal axis, basal-specific markers are displayed, such as \textit{KRT5} and the SERP family. \textit{AQP3, TP53, FGFR3} are highlighted in the scatter plot as markers known to be expressed across MIBC tumours.

% GSEA
The GSEA plot in \cref{fig:cs:ne_gsea} reveals several pathways related to collagen or the lagging strand, suggesting the involvement of these genes in Epithelial Mesenchymal Transition (EMT). This is emphasized by the GSEA run on the Hallmark database, presented in the Appendix \cref{fig:ap:cs:gsea_ne_hallmark}, which also identifies pathways involved in the cell cycle. The analysis of the NE quadrant highlights the neuronal nature of the subgroups but also underlines the potential to discover new genes specific to the subtype. 


% Summary
\section{Genes Summary} \label{s:cs:markers_summary}

The genes highlighted across the six MIBC subtypes in the previous subsection are summarised in the table \cref{tab:cs:genes_summaries}.


%%%%%% Neuroendocrine group %%%%%%
\begin{figure}[!htb]
    \centering
    \captionsetup{font=small} 
    \begin{subfigure}[!t]{1.0\textwidth}
        \includegraphics[width=\textwidth,keepaspectratio]{Sections/ClusteringAnalysis/Resources/discussion/other_groups/ne_pi.png}    
        \caption{Pi}
        \label{fig:cs:ne_pi}
    \end{subfigure}
    \centering
    \begin{subfigure}[!t]{0.9\textwidth}
        \includegraphics[width=\textwidth, keepaspectratio]{Sections/ClusteringAnalysis/Resources/discussion/other_groups/ne2_reactome_10_top.png}
        \caption{Top 10 GSEA results on the Reactome database.}
        \label{fig:cs:ne_gsea}
    \end{subfigure} 
    \centering
    \caption[Ne-like - molecular overview]{Molecular overview of the Neuroendocrine-like (NE) group. The Pi plot shows the genes specific to LumInf from the DEAs against Basal Low IFNG (X-axis) and Luminal Papillary (Y-axis). The points are coloured distinctly to represent the 98 \acrlong{tf} (Selected TF - next section \cref{s:N_I:sel_tfs}) and known markers presented in \cref{tab:ap:pi_genes_1,tab:ap:pi_genes_2}. Quadrant II highlights the genes specific to LumInf, which consists of new Neuronal markers and others new potential markers; the top 10 closest points to the 'Ref point' are displayed in orange. The GSEA plot displays the top 10 Reactome terms in the enrichment analysis, which primarily consist of cell-cycle pathways which is also found using the Hallmarks database in \cref{fig:ap:cs:gsea_ne_hallmark}.}
    \label{fig:cs:ne}
\end{figure}




\begin{table}[!htb]
  \centering
  \scriptsize
  \begin{tabularx}{\textwidth}{>{\hsize=0.8\hsize}X|>{\hsize=1.1\hsize}X|>{\hsize=1.1\hsize}X}
    \toprule
    \textbf{Group specific} & \textbf{Markers} & \textbf{Genes} \\
    \midrule
    \multicolumn{3}{c}{\textbf{Basal groups - \cref{fig:cs:pi_basal,fig:cs:basal_comp_zoom}}} \\
    \midrule    
    \multirow{3}{=}{\parbox[t]{\hsize}{\textbf{Low over High IFNG}}} & MIBC \cite{Robertson2017-mg} & \textit{TP53, DSC3} \\
    \cmidrule{2-3}
    & Luminal differentiation \cite{Robertson2017-mg, Ramal2024-ha}  &  \textit{ZBTB7C, FGFR3}\\
    \cmidrule{2-3}
    & Basal/Squamous \cite{Robertson2017-mg, Marzouka2018-ge, Kamoun2020-tj} & e.g. \textit{KRT} markers \\
    \midrule
    \multirow{2}{=}{\parbox[t]{\hsize}{\textbf{Low over Med IFNG}}} & Various - see \cref{s:cs:basal_interp} & \textit{DANCR, DAPL1, GATM, GSTM3, MLF1, SYNGR1, GSTM1, MAGI3} \\
    \cmidrule{2-3}
    & Basal/Squamous \cite{Robertson2017-mg, Marzouka2018-ge, Kamoun2020-tj} & \textit{DSC3, KRT6A, KRT5, TP53, and the SERPIN and SPR families} \\
    \midrule
    \multirow{2}{=}{\parbox[t]{\hsize}{\textbf{Both High \& Medium IFNG}}} & Basal/Squamous \cite{Robertson2017-mg, Marzouka2018-ge, Kamoun2020-tj} & \textit{ZBTB7C, ZNF750, KRT13, KLF5, SPRR1A} \\
    \cmidrule{2-3}
    & Immune and Immunoglobulin \cite{Baker2022-bj} & IG-family \\
    \midrule
    \multicolumn{3}{c}{\textbf{Ne - \cref{fig:cs:ne_pi}}} \\
    \midrule  
    \textbf{Ne} & Neuronal \cite{Robertson2017-mg}  & \textit{TUBB2B, GNG4, APLP1} \\
    \midrule
    \textbf{Ne over LumP} & Biological function see \cref{s:cs:ne_interp} & \textit{TUBA1A, FGFR1, SLC16A1, FXYD6, CTHRC1 and PRRX1} \\
    \midrule
    \multirow{2}{=}{\parbox[t]{\hsize}{\textbf{LumP specific over Ne}}} & Tissue differentiation \cite{Robertson2017-mg} & \textit{UPK2, GATA3, ADIRF, VSIG2} \\
    \cmidrule{2-3}
    & Various - see \cref{s:cs:lumP_interp} & \textit{KRT5, AQP3, TP53, FGFR3 and Serp} \\
    \midrule
    \multicolumn{3}{c}{\textbf{LumInf - \cref{fig:cs:lumInf_pi}}} \\
    \midrule  
    \textbf{LumInf over Low IFNG} & Luminal - \citep{Robertson2017-mg} & \textit{UPK1A, UPK3A, UPK2, PPARG, GATA3} \\
    \midrule
    \multirow{2}{=}{\parbox[t]{\hsize}{\textbf{Low IFNG over LumInf}}} & Basal - \citep{Robertson2017-mg} & \textit{KRT6A and KRT5} \\
    \cmidrule{2-3}
    & Infiltrated - see GSEA \cref{fig:cs:lumInf_gsea} & e.g. IG family \\
    \midrule
    \multicolumn{3}{c}{\textbf{LumP - \cref{fig:cs:lumP_pi}}} \\
    \midrule  
    \multirow{3}{*}{\textbf{LumP}} & Luminal - \cite{Robertson2017-mg} & \textit{PPARG, GATA3, UPK3A, FOXA1} \\
    \cmidrule{2-3}
    & Down-regulated \acrshort{scc} genes \cite{Hurst2022-sp} & See \cref{tab:ap:pi_genes_2} \\
    \cmidrule{2-3}
    & Various - see \cref{s:cs:lumP_interp} & \textit{VSIG2, PSCA, ADIRF, DHRS2, SPINK1} \\
    \midrule
    \textbf{Low IFNG over LumP} & Basal - \cite{Robertson2017-mg} & \textit{KRT6A, KRT6B, KRT14, KRT5} \\
    \midrule
    & \acrshort{scc} up-reg genes \citet{Hurst2022-sp} & See \cref{tab:ap:pi_genes_1}\\
    \bottomrule
  \end{tabularx}
  \caption[Summary of the genes found across the DEAs]{Summary of genes specific to the MIBC subgroups found in this section \cref{s:cs:bio_interp}. It uses a combination of Pi-plots, GSEA and known markers in the fields: TCGA markers \citet{Robertson2017-mg}, Consensus \citet{Kamoun2020-tj} and Lund \citet{Marzouka2018-ge}. The first column represents the groups to which the genes specific and with respect to which comparison; i.e. the first row contains the genes specific to the Low IFNG over the High IFNG. \textbf{Basal splits}: Low, Med, High \acrshort{ifn} (\cref{s:cs:basal_interp}), \textbf{Ne} to Neuroendocrine-like (\cref{s:cs:ne_interp}), \textbf{LumInf} - Luminal infiltrated (\cref{s:cs:lumInf_interp}) and LumP - Luminal Papillary (\cref{s:cs:lumP_interp}).   }
  \label{tab:cs:genes_summaries}
\end{table}


