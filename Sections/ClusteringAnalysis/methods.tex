

\subsection{Methods} \label{s:cs:methods}


% Quick overview
Generally, the clustering analysis can be split into several stages: 1) the data pre-processing 2) choosing the clustering model 3) choosing the number of K-means.

\paragraph*{Pre-processing} \label{s:cs:pre-processing}
% gene filtering
The RNAseq data consists of a large number of genes which are not always expressed across the samples. In this section, a gene is considered unexpressed if more than 10\% of the samples have a value less then 1.5 TPM. Filtering out the genes that meet the unexpressed conditions, leaves the TCGA's tumour dataset with $\sim13000$ genes that are expressed in at least 90\% of the samples (~370). The threshold of 10\% comes from the \citet{Robertson2017-mg} paper where the authors removed all the genes that have 'NA values more than 10\% across samples' (section \textit{Unsupervised mRNA expresseion clustering}). However, the authors were using RSEM data instead of TPMs.

From the resultant dataset a subset of the most varied genes are selected depending on the experiment. The most varied genes is taken by the highest standard deviation/median which also takes in account the expression's magnitude. Thus, this method of selection is choosing the highest relative varied genes across the samples. This is different from the method used by \citet{Robertson2017-mg} which selects the top 25\% only by the standard deviation and not factoring in the magnitude of the gene's expression.

\paragraph*{Finding the right configuration}

The following clustering models were explored for clustering analysi: K-means, Wards, Agglomerative Clustering and Spectral Clustering. These were introduced in \cref{s:lit:clustering} and the scikit-learn \cite{Scikit-learn_undated-ax} implementation was used. Note, there are other available clustering methods in the Python library but had poor performance in the dataset. 

To measure the performance the clustering metrics \cref{s:lit:clustering_metrics} were used: Silhouette with cosine distance, Calinski-Harabasz, and Davies-Bouldin. The elbow-method heurestic was also employed to determine the best performing model and number of clusters. 
